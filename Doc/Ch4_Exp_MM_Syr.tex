\chapter{Experimental Mode Matching Cavities at Syracuse}
In conjunction with Sandoval et al, the adaptive modematching table top experiment was able to show the feasibility of a fully dynamical system.
	\section{Adaptive Mode Matching}
	Real time digital system and model.
	
	Signal chain.
	
	\section{Actuators}
		\subsection{Thermal Lenses}
		Fabian's work and UFL paper.
		\subsection{Translation Stages}
		
	\section{Sensors}
		\subsection{Bullseye Photodiodes}
		As stated in section [], we saw that the error signal for a mode-mismatched cavity has cylindrical symmetry due to the beating between the LG-01 and the LG-00 mode.  This means that the quadrant photodiodes would not have a way to detect the modal content of a cavity.  One way to solve this is to introduce a type of detector that can sense the power outside versus the power inside (ugh re write this).
		
		In [Rana's IO final design document] 
		
		
		Why do we need bullseye photodiodes, meantion the geometry of the error signal.
		
		Derivations in the appendix.
		
		Picture of BPD
		
		Pitch and Yaw sensing matrix
		
		Demodulation phase
		
		Explain why we need $\omega_{0} = \sqrt{2} r_0$
		
		The ratio of out over in will give:
		
		\begin{equation}
		\text{Power Ratio} = \frac{\text{P}_2 + \text{P}_3 + \text{P}_4}{\text{P}_1}  \\
		= \frac{e^{-2r_0^2/ \omega_{0}^2}} {1 - e^{-2r_0^2/ \omega_{0}^2 }} \approx 0.582
		\end{equation}
		
		\subsection{Mode Converters}
		
		\subsection{Scanning Gaussian Beams}
		
		
		
