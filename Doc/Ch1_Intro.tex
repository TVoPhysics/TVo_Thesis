\chapter{Introduction}

	Say something profound here.
	
	Structure of this thesis:
	
		Gravitational waves and their detection
	
		The LIGO instrument and Noise + Squeezed States of Light
	
		Introduction to Wavefront Sensing
	
		Experimental Mode Matching at Syracuse
	
		Mode matching at LIGO Hanford
	
		Future Works

	\section{Gravitational Waves}\label{gravitational waves}
	In 1915, Albert Einstein published his theory of general relativity \ref{einstein}.
	
	The seminal equation in this theory is:
	
	\begin{equation} \label{einstein}
	G_{\mu \nu} = 8 \pi T_{\mu \nu}
	\end{equation}
	
	Which is a set of 10 coupled second-order differential equations that are nonlinear.  In its complete form, equation \ref{einstein} fully describes the interaction between space-time and mass-energy. 
	
	
	To describe the physics in a highly curved space-time, one would have to fully solve the Einstein field equations numerically. In areas where the curvature is close to flat,  the weak field approximation can be applied and the metric is described as	
	
	\begin{equation} \label{weak}
	g_{\mu \nu}  \approxeq \eta_{\mu \nu} + h_{\mu \nu}
	\end{equation}
	
	where $\eta_{\mu \nu}$ is the metric of flat space time and $|h_{\mu \nu}| \ll 1$ is the perturbation due to a gravitational field.
	
	By plugging in equation \ref{weak} into \ref{einstein} and using empty space we obtain the familiar wave equation
	
	\begin{equation} \label{wave}
	\Big(\nabla^2 - \frac{1}{c^2} \pdv[2]{t} \Big) h_{\mu \nu}  = 0
	\end{equation}

	which has a plane-wave solution of the form $h_{\mu \nu} = A_{\mu \nu} e^{ik_{\nu} x^{\nu}}$. 
	
	Using the gauge constraint $h^{\mu \nu}_{,\nu} = 0$, it follows that $A_{\mu \nu} k^{\mu} = 0$ which means that the gravitational wave amplitude is orthogonal to the propagation vector.
	
	Further imposing transverse-traceless gauge and assuming that the wave is traveling in the $x^3$ direction, it can be shown that the complex amplitude has physical significance expressed in the matrix
	
	\begin{equation} \label{gwamp}
	A_{\mu \nu} = 
	\begin{pmatrix}
			0 &    0   &  0      & 0 
		 \\ 0 & A_{xx} &  A_{yx} & 0
		 \\ 0 & A_{xy} & -A_{xx} & 0
		 \\ 0 &    0   &  0      & 0
	\end{pmatrix}
	\end{equation}

	Oftentimes, the four non-zero components of equation \ref{gwamp} can be categorized into two distinct polarizations called plus and cross such that $h_{+} = A_{xx} = -A_{yy}$ and $h_{\cross} = A_{xy} = A_{yx}$ .

 
	It is natural to attempt to understand the physical interpretation of equation \ref{gwamp} as an affect on the position of a free floating particle. Consider the four-velocity, $U^{\alpha}$, in the transverse traceless gauge where the coordinate itself is attached the particles and incorporates any small wiggles that would shake the coordinates.  Of course, any free particles will follow the geodesic equation
	
	\begin{equation}\label{geodesic}
	\nabla U^{\alpha} = \frac{\text{d}}{\text{d} \tau} U^{\alpha} + \Gamma^{\alpha}_{\mu \nu} U^{\mu} U^{\nu} = 0
	\end{equation}	
	where $\Gamma^{\alpha}_{\mu \nu} = \frac{1}{2} g^{\gamma \alpha}(g_{\gamma \mu, \nu} + g_{\gamma \nu,\mu} - g_{\mu \nu, \gamma} )$ are the famous Christoffel symbols.
	By evaluating the first term of the acceleration in equation \ref{geodesic},
	\begin{equation}\label{accel}
	\bigg(\frac{\text{d}U^{\alpha}}{\text{d}\tau}\bigg)_0 = -\Gamma^{\alpha}_{00} 
	\\ = \frac{1}{2} \eta_{\mu \nu} (h_{\beta 0, 0} + h_{0 \beta, 0} + h_{0 0, \beta} )
	\end{equation}
	However, comparing equation \ref{gwamp} and equation \ref{accel}, it is clear that if the particle is initially at rest, then a moment later it is still at rest! The term "at rest" is actually used liberally here since the coordinate system varies along with the gravitational wave. 
	
	Alternatively, one can ask if a gravitational wave passed by a pair of particles separated by length $L$, what would be the effect on the distance between two points?  The proper distance is defined as

	\begin{equation}\label{propdist}
	\delta l
	= \int{g_{\mu \nu} dx^{\mu} dx^{\nu}} \\
	= \int_{0}^{L}{g_{xx} {d}x}\\
	\approx |g_{xx}(x=0)|^{1/2}\\
	\approx [ 1 + \frac{1}{2} h_{xx}(x=0)] L
	\end{equation} 
	
	which shows us two very important points about the nature of gravitational waves.  Firstly, the effect is very small since the length variation, $h_{xx}$, is a small perturbation on flat space-time.  Secondly, the effect is proportional to the initial separation between the particles. This means a detector which is large will have a better chance to measure these small effects, an important point that drove the design of the Laser Interferometer Gravitational-Wave Interferometer (LIGO).
	
	\subsection{Measuring Gravitational Waves with Light}\label{measuringGWs}
	Even with the theoretical formulation of gravitational waves resolved by the 1970s ref[Pirani], detection of GWs by ground-based detectors was still a controversial topic among scientists in the field ref[Collins].  This was due to the incredible accuracy required to measure the strain from even the most dense astrophysical objects ref[Rai].  The earliest attempts at detecting these small signals were famously done by Joseph Weber ref[Weber] using large resonant bars and piezoelectric transducers to extract the energy from gravitational waves at the resonant frequencies of the bars. Picture of bar detectors at LHO.  However, these bars are limited by thermal noise and can only detect GWs in very narrow frequency bands ref[Bars]. 
	
	Interferometers are devices that measure small displacements by using a laser that is split with a partially transmitting mirror (or beamsplitter), which allows 50$\%$ of the light to get reflected and 50$\%$ to be transmitted.  Each of the split beams travel down arms and reflect off of mirrors and return down the arms.  Upon reaching the beamsplitter, the beams recombine and by using the superposition of electromagnetic waves the laser will add linearly at the output port (or antisymmetric port).  The laser beams will gather phase as they propagate down each individual arm, and when recombining, the intensity of the light will be proportional to the phase differences between each beam.  This will correspond to a differential length that is described by
	
	\begin{equation}
	L_{-} = l_{x} - l_{y}
	\end{equation}
	
	As shown in equation \ref{propdist}, the effect of gravitational waves on the proper length between two free falling objects is proportional to the initial separation.  From figure[GWparticles], it is intuitively clear that interferometry would be an ideal technique to detect signals from a gravitational wave.  However, one can explicitly derive a ground-based interferometer's response to a GW from an astrophysical object.
	
	Consider a gravitational wave source arbitrarily located in the sky with respect to an interferometer on Earth. By denoting the interferometer's Cartesian coordinates as $\{\hat{x},\hat{y},\hat{z}\}$ with x and y located along the arms respectively such that the z-axis points directly towards the zenith (i.e. a right-handed system.).  Using the well known Euler angles, a relation from the detector frame to the source frame with coordinates $\{\hat{x}',\hat{y}',\hat{z}'\}$ can be seen in Figure[Euler]. 
	
	If a gravitational wave at the source has emitted GWs with plus and cross polarizations as denoted by equation \ref{gwamp}, then the detector time series can be regarded as ref[Duncan Thesis and AndersonGWs]
	\begin{equation}
	h(t) = F_{+}(\theta,\phi,\psi) \, h_{+}(t) + F_{\times}(\theta,\phi,\psi) \, h_{\cross}(t)
	\end{equation}
	where $F_{+}(\theta,\phi,\psi)$ and $F_{\cross}(\theta,\phi,\psi$ are the antenna pattern functions that project the gravitational wave amplitudes onto the detector frame.
	
	\begin{equation}\label{Fplus}
	F_{+}(\theta,\phi,\psi) = -\frac{1}{2}[1+\text{cos}^2(\theta)] \text{cos}(2\phi) \text{cos}(2\psi) - \text{cos}(\theta) \text{sin}(2\phi) \text{sin}(2\psi)
	\end{equation}
	\begin{equation}\label{Fcross}
	F_{\cross}(\theta,\phi,\psi) = + \frac{1}{2}[1+\text{cos}^2(\theta)] \text{sin}(2\phi) \text{cos}(2\psi) - \text{cos}(\theta) \text{sin}(2\phi) \text{sin}(2\psi)
	\end{equation}
	
	If the gravitational wave is located directly above the interferometer (ie $\theta = 0$) and setting $\psi=0$, then the magnitude of the antenna pattern is equal to unity.  Furthermore, by rotating about the $\phi$ angle such that the detector arms align with the plus polarization, the null geodesic equation (ie the path of a photon) in the interferometer frame becomes 
	
	\begin{equation}
	ds^2 = g_{\mu\nu}dx^{\mu} dx^{\nu} = -dt^2 + [1+h_{+}]  dx^2 + [1-h_{+}]  dy^2 + dz^2 = 0
	\end{equation}
	
	Now if the photon is traveling along the x-arm, this means that $dy^2 = dz^2 = 0$ and the metric equation transforms to
	
	\begin{equation}
	\frac{dt}{dx} = \sqrt{ 1+h_{+} } \approx 1+\frac{1}{2} h_{+} 
	\end{equation}
	
	The amount of time required for the photon to reach the x-end mirror (starting at $t=0$) is equal to
	
	\begin{equation}\label{pathlength}
	t_1 = \int_{0}^{L_{x}} [1+\frac{1}{2}  h_{+}(x) ] \text{d}x
	\end{equation}
	
	where $t_0$ is the start time and $L_x$ is the total length of the x-arm.  Upon returning to the beamsplitter, the photon's total time of flight for the x and y arms are, respectively,
	
	\begin{equation}
	t_2 = 2 L_x + \frac{1}{2} \int_{0}^{L_x} \bigg[  h_{+}(x) +  h_{+}(x + L_x)  \bigg] \text{d}x
	\end{equation}
	\begin{equation}
	t'_{2}= 2 L_y - \frac{1}{2} \int_{0}^{L_y} \bigg[  h_{+}(y) +  h_{+}(y + L_y)  \bigg] \text{d}y
	\end{equation}
	
	If the gravitational wave period is much longer than the time of flight, then $h_{+}$ does not change much during the measurement, which means $h_{+}(\eta_i) \approx h_{+}(\eta_i + L_{\eta_i}) \approx constant$.  By subtracting the flight times of the photons for each arm and setting $L = L_x = L_y$, the difference is proportional to the gravitational wave perturbation multiplied by the sum of arm lengths (with a factor of $c$ to get the units right),
	
	\begin{equation}
	\Delta t = t_2 - t'_{2} = \frac{2L}{c} h_{+}
	\end{equation}
	
	By recasting the expression for time of flight in terms of the phase picked up laser light as it travels through space, the differential phase shift is
	\begin{equation}\label{diffphase}
	\Delta \Phi = \Phi(t_{2}) - \Phi(t'_{2}) = \frac{4 \pi}{\lambda} \, h_{+} \, L
	\end{equation}
	
	The equation above is simple, however, it only works for gravitational wave signals that are not frequency dependent and it assumes that the path length can be arbitrarily long. Both points are actually not true but we can alleviate these discrepancies by considering a gravitational wave signal of the form $h(t) = h_0 \text{exp} (i 2 \pi f_{GW} t)$  and repeating the calculation between equations \ref{pathlength} and \ref{diffphase}:
	
	\begin{equation}\label{gwsinc}
	\Delta \Phi (t) = h(t) \; \tau \; \frac{2 \pi c}{\lambda} \; \text{sinc}(f_{GW} \tau) e^{i \pi f_{GW} \tau}
	\end{equation}
	
	where $\tau_{RT} = 2L/c$.  The response for a detector whose arm lengths are blank km have null points at the around blank Hz which means the instrument cannot be arbitrarily long, however, this point is not concerning because it is too expensive and difficult to make a terrestrial detector of this size.
	
	How to practically measure $\Delta \Phi$ with an interferometer is explained in section \ref{LIGOInstrument}.

	\subsection{Detection of Gravitational Waves}
	The LIGO project is meant to observe astrophysical objects so it is natural to wonder how well a single detector can probe the universe.  
	
	Signal-to-noise is the level of interesting signals compared to the background noise, which can be expressed as
	\begin{equation}\label{SNR}
	\rho = 2 \int_{-\infty}^{\infty} \frac{ \tilde{s}(f) \tilde{s}^*(f) }{S_n(f)}
	\end{equation}
	
	where ${S_n(f)}$ is the one-sided average power spectral density of the detector noise and $\tilde{s}(f)$ is the Fourier transform of the detectors' response to a gravitational-wave signal.
	
	Consider two dense objects with mass $m_1$ and $m_2$ rotating around each other, separated by a distance $R$ such that quadrupole equation still holds (i.e. flat space-time with small perturbations). The detector response, in units of strain, to the gravitational waves emitted by this source is [Finn],
	
	\begin{equation}\label{inspiralsignal}
	s(t) = \frac{\mathcal{M}}{d_L} \Theta(\theta,\phi,\psi,i) [\pi f(t) \mathcal{M}]^{2/3} \cos[\Phi(t) + constant]
	\end{equation}
	
	where $\mathcal{M} = (1+z) \frac{(m_1 m_2)^{3/5}}{(m_1 + m_2)^{1/5}}$ is the chirp mass, $d_L$ is the luminosity distance. 
	
	
	The orientation response, $\Theta(\theta,\phi,\psi,i)$, is a function that depends entirely on the detector orientation relative to the angular momentum vector of the binary [Figure of a binary spiraling and a detector on earth]
	
	\begin{equation}
	\Theta(\theta,\phi,\psi,i) = 2 \sqrt{	[F_{+}(\theta,\phi,\psi) (1+\cos^2(i)) ]^2 + [2 F_{\cross} \cos(i)]^2 }
	\end{equation}
	
	where $F_{+}$ and $F_{\cross}$ are from equation \ref{Fplus} and \ref{Fcross}.
	
	As the binary loses energy to gravitational waves, the orbit will shrink [ref Schutz or something] as a function of time, in turn, the orbital frequency will increase
	
	\begin{equation}
	f(t) = \frac{1}{\pi \mathcal{M}} \bigg[\frac{5}{256} \frac{\mathcal{M}}{T-t}\bigg]^{3/8} 
	\end{equation}
	
	By defining the binary phase as $f(t) = \frac{1}{2\pi} \frac{\partial \mathbf{\Phi} }{\partial t}$, it is easy to recognize that the phase also evolves with time.
	\begin{equation}
	\Phi(t) = 2\pi \int_{T}^{t} f(t) dt = -2 \bigg( \frac{T-t}{5\mathcal{M}}\bigg)^{5/8}
	\end{equation}
	
	What this means is that the gravitational wave signal from the binary will increase in amplitude and frequency as a function of time up until the coalescence time, $T$.


	There is actually a subtle difference between the coalescence time and the moment when adiabatic approximations fail which allows usage of the quadrupole formulation. 

	Plugging equation \ref{inspiralsignal} into \ref{SNR},
	\begin{equation}
	\rho = 8 \Theta(\theta,\phi,\psi,i) \frac{r_0}{d_L} \bigg(\frac{\mathcal{M}}{1.2 \textup{M}_\odot}\bigg)^{5/6} \zeta(f_{max})
	\end{equation}
	
	There are two important functions in the equation above that reflect the detector's performance in sensing gravitational radiation from a binary inspiral, $r_0$ and $\zeta(f_{max})$.
	
	The characteristic distance, $r_0$ is how far the detector can see for a fixed binary's mass distribution,
	
	\begin{equation}\label{char_r0}
	r_0^2 = \frac{5}{192\pi^{4/3}} \bigg(\frac{3\textup{M}_\odot}{20}\bigg)^{5/3}   \int_{0}^{\infty} \frac{1}{S_n(f)} \frac{\text{d} f}{f^{7/3}}
	\end{equation}

	Due to the integrand's dependence on $f^{-7/3}$, lower frequency improvements in the detector's noise spectrum will contribute more to the distance.
	
	\begin{equation}\label{zeta}
	\zeta(f_{max}) = \frac{\int_{0}^{2f_{max}} \frac{1}{S_n(f)} \frac{\text{d} f}{f^{7/3}}}{\int_{0}^{\infty} \frac{1}{S_n(f)} \frac{\text{d} f}{f^{7/3}}}
	\end{equation}
	
	The sensitivity can be different for various mass distributions; $\zeta(f_{max})$ is a normalized function that describes how well the detector's bandwidth overlaps the binary's frequency evolution.  If $f_{max}$ is higher that the minimal sensitivity frequency for the detector, then there is good overlap and  $\zeta(f_{max})\approx 1$.  However, if the masses are sufficiently large, $f_{max}$ will be lower because coalescence occurs before the objects reach a higher frequency regime and this will result in  $\zeta(f_{max})\approx 0$.
	
	The process of two objects coalescing starts at lower frequencies and terminate at some $f_{max}$ that depends on when the objects reach their inner-most stable circular orbit, $f_{ISCO}$.   If $m_1=m_2$, then the maximum frequency is[]
	
	\begin{equation}
	\begin{aligned}
	f_{max}	=&  \frac{f_{ISCO}}{1+z} \\
			=&	\frac{99 \text{Hz}}{1+z} \, \frac{20 \textup{M}_\odot}{M}
	\end{aligned}
	\end{equation}
	where $M=m_1+m_2$.
	
	
	By plugging a few mass distributions and an estimate of the sensitivity [GWINC], it is possible to study the range of a single detector.  For example, a 1.4-1.4$\textup{M}_\odot$ binary that has optimal orientation with respect to the detector has horizon distance equal to...
	
	Horizon distance and range, advanced LIGO projected
	
	[Finn, Duncan Thesis]
	\cite{Saulson}
	
	\section{The LIGO Instrument}\label{LIGOInstrument}
	In its simplest form, the LIGO instrument is an incredibly large Michaelson interferometer.  If we imagine the output as a measure of differential arm lengths, it becomes a natural way of detecting gravitational waves. However, to make a practical gravitational-wave observatory, the complexity will have to be extended beyond what Michelson and Morley used.  The next sections will explore the various upgrades that were implemented to improve the sensitivity of LIGO.
	
		\subsection{Simple Michaelson}\label{michelson}
		As shown in Figure [michelsonifo], the interferometer readout uses a photodetector that measures the total laser beam power and depends on how light in the arms constructively or destructively interferes at the beamsplitter. In Section \ref{measuringGWs}, it was shown that the differential time of flight between photons traveling down the individual arms carry gravitational wave information.  The difference in flight times, $\Delta t$, can be converted into how much phase,$\Delta \phi$ is accumulated by the photons as they propagate through space.  But the question remains how an interferometer actually measures $\Delta \phi$.
		
		If the input electric field of the interferometer is $E_0$, the beamsplitter will transmit $E_0 /2$ down the x-arm and reflect $E_0 /2$ down the y-arm.  By setting the beamsplitter to be the origin,  two beams traveling down their respective arms will have gathered a phase $\phi_i$. Then, upon reflecting off the end mirrors and returning to the beamsplitter, each of the electric fields can be described by these equations
			\begin{equation}
			\begin{aligned}
				E_{x} 	&=	\frac{i E_0}{2} e^{2i\phi_{x}}	
			\\	E_{y} 	&=	\frac{i E_0}{2} e^{2i\phi_{y}}
			\end{aligned}
			\end{equation}
		Since the electromagnetic waves are linear, the resultant sum of waves at the output will be $E_{out} = E_x + E_y$. A photodiode (PD) is placed at the output (or antisymmetric) port to read out the integrated power which is related to the total electric field by
			\begin{equation}
			\begin{aligned}\label{asy_power}
				P_{AS}	&= \int_{Area} I \;				\text{d}A 
			\\			&= \int_{Area} \vert E_x + E_y \vert^2 \;	\text{d}A 
			\\			&= P_{in} \; \text{cos}^2(\Delta \phi)
			\end{aligned}
			\end{equation}	
		where $\Delta \phi = \phi_{x} - \phi_{y} = k_x L_x - k_y L_y$ and $P_{in}$ is the input power. By using equation \ref{diffphase} and the common (or average) arm length $L_{+} = \frac{L_x + L_y}{2}$, the power due to a differential phase shift is
			\begin{equation}
			P_{AS} \approx P_{in} \; (1-2 \Delta \phi) = P_{in} \; (1-2 k L_{+} h_{+})
			\end{equation}
		There is a large DC term that is dependent on the input power and generally, it is very difficult to measure small changes in a large signal. So the next obvious method would be to shift the arms such that the output port is operating on a dark fringe, normally this is called a null-point operation.  However, there are difficulties associated with this method as well.  
		Consider shifting the phase of equation \ref{asy_power} by $\pi/2$, which would result in
			\begin{equation}\label{null}
			P_{AS} \vert_{null} = P_{in} \; \text{sin}^2 (\Delta \phi) \approx P_{in} \; (k L_{+} h_{+})^2 
			\end{equation}
		This results in a second-order dependence on a gravitational-wave signal that is already approximated to be very small.  
		So a good solution to the issue of how to $read$ out a gravitational wave signal can be solved using radio frequency (RF) detection methods. Consider changing the interferometer input by adding an electro-optical modulator (EOM) to sinusoidally modulate the laser frequency and expanding to first order using the Bessel functions,
			\begin{equation}\label{modE}
			\begin{aligned}
			E_{in} 	&= E_{0} e^{i(wt + \beta \text{cos} (\Omega t))} \\
					&\approx E_0 e^{iwt} [J_0(\beta) + J_1(\beta) e^{i \Omega t} + J_1(\beta) e^{-i \Omega t}] \\
					&= E_{C,in} + E_{SB+,in} + E_{SB-,in}
			\end{aligned}
			\end{equation}
		where $\Omega$ and $\beta$ are the modulation frequency and depth, respectively. The first term is commonly called the carrier field whereas the second and third terms are referred to as the (upper or lower) sidebands.  Because there are multiple electric fields, it is useful to define an optical transfer function which transforms the interferometer's input fields to its output,
		\begin{equation}\label{opt_tf}
		E_{out} = E_{C} + E_{SB+} + E_{SB-} = 
		\begin{pmatrix}
			t_{C} 	&   
		\\ 	t_{SB+} &
		\\ 	t_{SB-} &
		\end{pmatrix}
		\begin{pmatrix}
		E_{C,in} &    E_{SB+,in}    &  E_{SB-,in}     
		\end{pmatrix}
		\end{equation}
		The carrier transfer function, $t_{C}$ has already been calculated by equations \ref{asy_power} - \ref{null} and the sideband transfer functions are not much different.
		\begin{equation}\label{sb_tf}
		t_{SB\pm} = r_{x,\pm}  e^{i\phi_{\pm,x}} - r_{y,\pm}  e^{i\phi_{\pm,y}}
		\end{equation}
		where $\phi_{\pm,i} = (k \pm k_{\Omega}) \, \ell_{i} = (\frac{w+\Omega}{c} ) \ell_{i}$. In the current example, the sidebands and carrier fields reflect off the end mirrors identically, however, this will not be true in general when dealing with resonators that are highly frequency dependent.  Plugging equation \ref{sb_tf} into \ref{opt_tf}, the output electric field becomes 
		\begin{equation}
		E_{out} = i e^{iwt} [ J_0(\beta) 	k \ell_{+}  h_{+}  \; + \; J_1(\beta) \sin( k \Delta \ell + k_{\Omega} \Delta \ell) (e^{i\Omega t}  + e^{-i\Omega t}) ]
		\end{equation}
		By choosing the carrier signal to be on the dark fringe, $k \Delta \ell = \pi/2$ but the sidebands to be slightly off the null (Figure []) and hence leak into the anti-symmetric port, the electric field reduces to
		\begin{equation}
		E_{out} = i e^{iwt} [ J_0(\beta) 	k \ell_{+}  h_{+}  \; + \; J_1(\beta) \sin(k_{\Omega} \Delta \ell) ( e^{i\Omega t} + e^{-i\Omega t}) ]
		\end{equation}
		
		Recall that the intensity is equal to the electric field squared,
		\begin{equation}\label{RFdet}
		\begin{aligned}
			I	= \vert E_{out} \vert^2  =	&\vert E_{C}\vert^2 + \vert E_{SB+}\vert^2 + \vert E_{SB-}\vert^2 \\
										  	& + 2 \mathbf{Re} \{ \; E_{SB+} E^*_{SB-} e^{2i\Omega t} \; \}\\
										  	& + 2 \mathbf{Re} \{ \; (E_{C} E^*_{SB-} +  E_{SB+} E^*_{C} ) e^{i\Omega t} \; \}
		\end{aligned}
		\end{equation}
		The last term is referred to as the $beat$ $note$ between the carrier signal and the sidebands.  It is possible to extract the term at the modulation frequency using a mixer which is an analog device that outputs the product of two inputs. Usually, the same oscillator that was used to modulate the input beam can be one of the mixer inputs, $\cos(\Omega t)$,  so that the demodulated signal is
		\begin{equation}
		\begin{aligned}
		I_{Demod} 	&\propto \big[ 4 \pi  J_0(\beta) J_1(\beta) \frac{\ell}{\lambda}  \sin(k_{\Omega} \Delta \ell)  \; h_{+}\big] \big[ \cos(\Omega t)  \sin(\Omega t + \phi_{Demod}) \big] \\
					&= \big[ 4 \pi  J_0(\beta) J_1(\beta) \frac{\ell}{\lambda}  \sin(k_{\Omega} \Delta \ell)  \; h_{+}\big] \big[ \sin(\phi_{Demod}) + \sin(2\Omega t + \phi_{Demod}) \big]
		\end{aligned}
		\end{equation}
		where $\phi_{Demod}$ is the phase that can be set by the user in order to account for extra phase shifts (ie. longer cables). After the mixer, there will be signals at DC, $\Omega$, $2\Omega$ and so on. However, the part that is linear in the gravitational wave amplitude will be at DC so a low-pass filter will allow the final signal to dominate:
		\begin{equation}
		S = 4 \pi  J_0(\beta) J_1(\beta) \frac{\ell}{\lambda}  \sin(k_{\Omega} \Delta \ell) \sin(\phi_{Demod}) \; h_{+}
		\end{equation}
		This shows that a RF detection technique will be linear in GW signal with no large DC offset. Setting the carrier on a null point means $\Delta \ell = \frac{k_{\Omega}}{k} \frac{\pi}{2}$ and allows the designer to optimize the Schnupp asymmetry length to get the best signal for some modulation frequency. This type of readout scheme was used in Enhanced LIGO and is called heterodyne detection, where the sideband fields are produced by an EOM and its efficacy depends on the local oscillator's stability \cite{FritschelReadout}.  
		In contrast, the Advanced LIGO scheme uses a homodyne detection \cite{HildDCReadout} method called "DC-Readout".  Here the oscillator field is produced by slightly offsetting the arms away from the dark fringe and letting a small amount of carrier light through the antisymmetric port.  A gravitational wave will induce sidebands on the carrier and this will allow the same mathematics as above to achieve a linear signal in gravitational wave strain. This method benefits from naturally being co-aligned and mode matched with the signal field.  All techniques follow the same logic of beating the field containing useful information with a reference field to extrapolate a linear signal but the differences come from technical noise such as laser intensity fluctuations and effective quantum noise.
		
		\cite{BlackPDH}	
	
		\subsection{Fabry-Perot Cavities}\label{FP}
		There are two ways to improve the LIGO detectors: one is to increase the response from gravitational waves and the other is to decrease the noise contributions. From equation \ref{diffphase}, the gravitational wave signal is proportional to the optical path length that the photon travels, which means the most straightforward method of increasing the sensitivity is to make the arms as long as possible (up to the null point described by equation \ref{gwsinc}.  Generally, there were two methods to do this: a Herriott delay line or a Fabry-Perot resonator, the differences between each method is shown in Figure[].  At the time of writing this thesis, all modern gravitational wave detectors use the latter method.
	
		A Fabry-Perot cavity is an optical system comprised of two or more partially transmitting mirrors with one laser input.  To create a resonator, the user must design a system such that once the laser has made one round trip around the optics, it is the same shape and size as when it started.  Conceptually, this may seem simple but in practice, controlling and sensing any optical cavity comes with a few challenges.
		
		To start understanding the longitudinal degree of freedom, consider a two mirror system in Figure [] which is separated by a length $L$ with reflection and transmission coefficients: $r_1$, $t_1$, $r_2$, $t_2$.	
		Starting with a plane wave at the input mirror with amplitude $E_0$, the beam will enter the cavity add on top of each other such that the reflected field \cite{Saulson} is 
		\begin{equation}\label{r_FP}
		E_{REFL} = r_{FP} E_0 = \bigg(-r_1 + \frac{t_1^2 r_2  e^{-i2kL}}{1-r_1 r_2 e^{-i2kL}} \bigg) E_0,
		\end{equation}
		the transmission field is
		\begin{equation}\label{t_FP}
		E_{TRAN} = t_{FP} E_{0} = \bigg( \frac{t_1 t_2 e^{ikL}}{1-r_1 r_2 e^{-i2kL}}\bigg) E_0
		\end{equation}
		the circulating field is
		\begin{equation}\label{c_FP}
		E_{CIRC} = c_{FP} E_0 = \bigg(\frac{t_1}{1- r_1 r_2 e^{-2ikL}} \bigg) E_0
		\end{equation}
		The fields become resonant when the cavity length is $L = n \lambda / 2$ and the circulating coefficient in the cavity is maximized such that the gain is
		\begin{equation}
		\text{Gain} = c^2_{FP} \vert_{\text{resonating}} = \bigg( \frac{t_1}{1-r_1 r_2}\bigg)^2
		\end{equation}
		Depending on the relative reflection coefficients of the input and output mirrors, the fields on resonance will be slightly different Figure[].
		
		Frequency response of a single FP (plot):
		
		Notice that the circulating power dependent on the cavity length and laser frequency so one might naively determine that modulating the two parameters independently cause the same effect.  However, when both are changing by large amounts, they are related by a frequency dependent transfer function
		\begin{equation}
		C(s) \frac{\Delta w}{w} = -\frac{\Delta L}{L}
		\end{equation}
		where $C(s) = \frac{1-e^{-2sL/c}}{2sL/c}$ in the Laplace domain.  Only when the cavity is on or near resonance, then the frequency and length variations are related by $\frac{\Delta w}{w} = -\frac{\Delta L}{L}$.
		
		While sweeping through either laser frequency or cavity length and measuring the reflected (or transmitted) fields, there are features of the power spectrum which relate directly to the cavity's physical properties:
		
		Finesse, or the line width of the resonant peak, function of $r_1$ and $r_2$:
		\begin{equation}
		\mathbb{F} = \frac{\pi \sqrt{r_1 r_2}}{1- r_1 r_2}
		\end{equation}
		
		Storage Time :
		\begin{equation}
		\tau_{s} = \frac{L}{c \pi} \, \mathbb{F}
		\end{equation}
		
		Cavity Pole:
		\begin{equation}
		f_{pole} = \frac{1}{4\pi \tau_{s}}
		\end{equation}
	
		Free Spectral Range:
		\begin{equation}
		f_{FSR}  = \frac{c}{2L}
		\end{equation}
		
		Stability: In order to prove that the Fabry Perot is stable, it is useful to introduce the matrices that describe a periodic optical system which is explicitly derived in appendix \ref{FPappendix}.  A Fabry Perot cavity that is separated by distance $L$ with spherical mirrors that have radii of curvature $R_1$ and $R_2$ will need to satisfy 
		\begin{equation}\label{gfactor}
		0 \geq \bigg(1+\frac{L}{R_1}\bigg) \bigg(1+\frac{L}{R_2}\bigg) \geq 1
		\end{equation}
		in order to be geometrically stable.
		
		Power circulating as a function of our defined parameters slightly off resonance by a length of $\delta L$:
		\begin{equation}
		P_{cav} = \vert c_{FP} \vert^2 = \frac{Gain}{ 1 + \big(\frac{2\mathbb{F}}{\pi} \big)^2 \, \text{sin}^2(k \delta L) }
		\end{equation}
		\subsubsection{Locking a Fabry Perot Cavity}
		This is described everywhere, reference \cite{BlackPDH}, Drever. Just give the highlights.
		
		Described above are the theoretical constructs of a FP cavity, but the question remains, how does one practically construct a resonant optical cavity?  The answer comes from using a heterodyne sensing scheme similar to the one described in Section \ref{michelson}.  Except the optical system is not a Michelson interferometer but rather a two mirror cavity, however, heterodyne detection can apply to a number of different geometries such as triangular or bow-tie cavities shown in Figure [].  All of which are used in LIGO for various reasons.
		
		Starting with an input laser and EOM (electro-optical modulator) that imparts upper and lower sidebands at a modulation frequency $\Omega$, the user injects three beams into the optical system exactly as in Equation \ref{modE}.  When placing a photodetector on the reflection port, one should see the cavity's effect on each of the three electric fields.
		
		\begin{equation}
		E_{FP,out} = E_{C} + E_{SB+} + E_{SB-} = 
		\begin{pmatrix}
		r_{C} 	&   
		\\ 	r_{SB+} &
		\\ 	r_{SB-} &
		\end{pmatrix}
		\begin{pmatrix}
		E_{C,in} &    E_{SB+,in}    &  E_{SB-,in}     
		\end{pmatrix}
		\end{equation}
		
		where the reflection coefficients follow equation \ref{r_FP}.  Because sidebands are frequency shifted, they will effectively $see$ a different cavity than the carrier and the total phase accumulated between the fields will be different. To make a good reference for the resonant carrier beam, the modulation frequency, $k_{\Omega}$, is chosen such that sidebands are be anti-resonant in the cavity.  
		\begin{equation}
		r_{C} = -r_1 + \frac{t_1^2 r_2  e^{-i2kL}}{1-r_1 r_2 e^{-i2kL}}
		\end{equation}
		and 	
		\begin{equation}
		r_{SB\pm} = -r_1 + \frac{t_1^2 r_2  e^{-i2(k+k_{\Omega})L}}{1-r_1 r_2 e^{-i2(k+k_{\Omega})L}}
		\end{equation}
		The formalism to read out the error signal is the shown in Equation \ref{RFdet}. Using a photodetector in reflection, the error signal will be linearly proportional to the laser frequency and cavity length \cite{BlackPDH}.
		\begin{equation}
		\text{Error Signal} \propto \frac{L \mathbb{F}}{\lambda} \bigg[\frac{\delta w}{w} + \frac{\delta L}{L}\bigg]
		\end{equation}

		\subsubsection{Application to LIGO}
		
		Frequency response of a FP Michaelson:
		Just write it down.
		References:[\cite{BlackSignalExtraction}]

		\subsection{Power-Recycled Fabry-Perot Interferometers}
		Power Recycling
		If the interferometer is operating such that the 4 km arms are exactly different in arms a pi over two times the wavelength, then the intensity of the light at antisymmetric port will be close to null.  This means the power from the arms will reflect back towards the input laser.  Ref[] shows the effect of adding a partially reflecting mirror to increase the optical gain of the Michaelson for the sidebands and carrier fields.
		
		
		
		References: [Meers, Kiwamu]
		
		\subsection{Dual-Recycled Fabry-Perot Interferometers}
		
		One of the biggest changes made between initial and advanced LIGO was the addition of a signal recycling mirror at the antisymmetric port shown in Figure []. This extra mirror allows an extra degree of freedom in shaping the sensitivity curve such.
		
		
		Figure comparing the three cases of increased sensitivity.
		
		References: [Kiwamu]
		
		\subsection{Fundamental Noise Sources}
		Ref: Evan Hall, GWINC
		The proceeding sections describe ways to increase the response of LIGO to gravitational waves; equally as important is the science of characterizing and reducing the noise contributions from everything else.
		
		Noise budget:
		
		\subsubsection{Quantum Noise}

		One fundamental noise source that is limiting LIGO's sensitivity comes from the fluctuations of quantum vacuum entering the anti-symmetric port and coupling to the input laser. A quantum mechanical description of an interferometer was constructed by Caves \cite{CavesQMNoise}\cite{Caves2photon} \cite{CavesOscillator}, where he used electric field operators to show that vacuum fluctuations are the cause of radiation pressure and shot noise in an interferometer.
		
		\textbf{Quantum states}:
		It is well known in physics \cite{Shankar} \cite{Griffiths} that a solution to the quantum harmonic oscillator in the energy eigenbasis employs the annihilation and creation  operators, $\hat{a}^{\dagger}$ and $\hat{a}$, to factorize the Hamiltonian
		\begin{equation}
		\hat{H} = \hbar w (\hat{N} + 1/2)
		\end{equation} 
		where $\hat{N} = \hat{a}^{\dagger}  \hat{a}$ is the number operator.  When using this formalism to create a coherent electromagnetic field, it is useful to define a unitary operator that displaces the vacuum state \cite{GerryKnight}:
		\begin{equation}
		\hat{D} \equiv \hat{D}(\alpha) \equiv \text{exp}(\alpha \hat{a}^{\dagger} - \alpha^{*} \hat{a} ) = e^{-\frac{\vert{\alpha}\vert^2}{2}} e^{\alpha \hat{a}^{\dagger} } e^{\alpha^{*} \hat{a} }
		\end{equation}
		$\hat{D}(\alpha) = \hat{D}^{-1}(\alpha) = \hat{D}(-\alpha)$
		\begin{equation}
		\begin{aligned}
		\hat{D}^\dagger&\, \hat{a} 		\,\hat{D}			= \hat{a} + \alpha \\ 
		\hat{D}^\dagger&\, \hat{a}^\dagger \,\hat{D} 		= \hat{a}^\dagger + \alpha^*
		\end{aligned}
		\end{equation}
		\begin{equation}
		\ket{\alpha} = \hat{D} \ket{0} =  e^{-\frac{\vert{\alpha}\vert^2}{2}} e^{\alpha \hat{a}^{\dagger} } \ket{0}
		\end{equation}
		
		
		\textbf{Radiation Pressure}
		
		One might naively think that power fluctuations in the laser cause radiation pressure effects on the test masses which will result in noise.  However, if the 50/50 beamsplitter is perfect, then the momentum transfer to each test mass will be a common length change and will not vary the intensity at the antisymmetric port (or the symmetric port for that matter).
		
		The concept of quantum radiation pressure arises from considering plane wave waves entering the interferometer from both the symmetric and anti-symmetric ports.  This method is similar to the input-output methods of section \ref{michelson}, however, the difference being that the beamsplitter will couple the electric fields from the input laser and quantum vacuum.
		
		To start, consider the electric fields combining at the beamsplitter from both ports to strike the mirrors, respectively,
		\begin{subequations}\label{exey}
		\begin{equation}
		E_x = \frac{1}{\sqrt{2}} \bigg[ iE_0 +   E_{AS,in} \bigg]
		\end{equation}
		\begin{equation}
		E_y = \frac{1}{\sqrt{2}} \bigg[  E_0 + i E_{AS,in} \bigg]
		\end{equation}
		\end{subequations}
		The intensities hitting each test mass will be
		\begin{subequations}
		\begin{equation}
		\vert E_x \vert^2 = \frac{1}{2} \bigg[ \vert E_0 \vert^2 + \vert E_{AS,in} \vert^2  + i(E_0 E^*_{AS,in} - E_0^*E_{AS,in} )\bigg]
		\end{equation}
		\begin{equation}
		\vert E_y \vert^2 = \frac{1}{2} \bigg[ \vert E_0 \vert^2 + \vert E_{AS,in} \vert^2  - i(E_0 E^*_{AS,in} - E_0^*E_{AS,in} )\bigg]`
		\end{equation}
		\end{subequations}
		
		The differential momentum transfer between the two masses will be equal to the differences in intensities:
		\begin{equation}\label{momentumtranfer}
		\begin{aligned}
		 \mathbf{P} 	&= \frac{2 \hbar \omega}{c} \bigg( \vert E_x \vert^2 - \vert E_y \vert^2 \bigg) \\
						&= \frac{2 \hbar \omega}{c} \bigg( E_0 E^*_{AS,in} - E_0^*E_{AS,in} \bigg)\\
		\Rightarrow	\mathbf{\hat{P}}&= \frac{2 \hbar \omega}{c} \bigg( \hat{a}_1^{\dagger} \hat{a}_2 - \hat{a}_2^{\dagger} \hat{a}_1 \bigg)
		\end{aligned}
		\end{equation}
		The last part of equation \ref{momentumtranfer} replaces the classical electric fields with creation and annihilation operators for the symmetric input mode ($\hat{a}_1^{\dagger}$, $\hat{a}_1$) and antisymmetric input mode ($\hat{a}_2^{\dagger}$, $\hat{a}_2$) modes. Recall that there is a well established convention denoting the wave function of the two dimensional modes of the quantum harmonic oscillator \cite{GerryKnight}, $\ket{\alpha,\beta}$, where $\alpha$ denotes the input symmetric mode and $\beta$ refers to the input antisymmetric mode.
		\begin{equation}
		\ket{\alpha,0} = \hat{D}_1(\alpha) \ket{0,0}
		\end{equation}
		The interesting results arising from this formulation is the expectation value
		\begin{equation}
		\langle \mathbf{\hat{P}} \rangle = \bra{\alpha,0} \mathbf{\hat{P}}\ket{\alpha,0} = 0
		\end{equation}
		And the variance 
		\begin{equation} 
		\begin{aligned}
		\Delta \mathbf{P}^2 &= \langle \mathbf{\hat{P}}^2 \rangle  - \langle \mathbf{\hat{P}} \rangle^2 \\
							&= \bigg(\frac{2 \hbar \omega}{c}\bigg)^2 \bra{\alpha,0}  \hat{a}_1^{\dagger}  \hat{a}_2  \hat{a}_1  \hat{a}_2^{\dagger} + \hat{a}_2^{\dagger}  \hat{a}_1  \hat{a}_2  \hat{a}_1^{\dagger}
							- \hat{a}_2^{\dagger}  \hat{a}_1  \hat{a}_1  \hat{a}_2^{\dagger} - \hat{a}_1^{\dagger}  \hat{a}_2  \hat{a}_2  \hat{a}_1^{\dagger} \ket{\alpha,0}\\
							&= \bigg(\frac{2 \hbar \omega}{c}\bigg)^2 \vert \alpha \vert^2
		\end{aligned}
		\end{equation}
		For some time interval, input laser power consisting of $\langle N \rangle =\vert\alpha \vert^2$ photons for a time interval $\Delta T$ is,
		\begin{equation}
		P_{in} = \frac{\hbar \omega}{\Delta T} \vert \alpha \vert^2
		\end{equation}
		To solve for the amplitude spectral density of the displacement, Newton's second law can be applied in the frequency domain
		\begin{equation}
		M (2\pi f)^2 \tilde{x}(f) = \tilde{F}(f) = \frac{\Delta \mathbf{P}}{\Delta T}
		\end{equation}
		The force spectrum is white because the impacting photons arrive randomly.
		
		Solving for the displacement,
		\begin{equation}
		\tilde{x}_{RP}(f) = \sqrt{\frac{\hbar \omega}{\Delta T} P_{in}} \frac{1}{2Mc (\pi f)^2}
		\end{equation}
	
		The noise spectral density for $M=40kg$, $P_{in}=125$W, and $\lambda=1064$nm
		\begin{equation}
		\tilde{h}_{RP}(f) = 2.04\times 10^{-20} \frac{1}{f^2} \; \bigg[ \frac{\text{Strain}}{\sqrt{\text{Hz}}}\bigg]
		\end{equation}
		\textbf{Shot Noise}
		Another way that quantum fluctuations can vary the antisymmetric port output is by adding phase noise.  Imagine holding the test masses rigidly such that the only effects on the output light is due to a phase change in the laser light in the arms. By using the same formulation for radiation pressure but propagating the fields back to the beam splitter, equations \ref{exey} will have extra phase,
		
		\begin{subequations}\label{exeyout}
		\begin{equation}
		E_{x,out} = \frac{1}{\sqrt{2}} \bigg[ iE_0 +   E_{AS,in} \bigg] e^{-i2kL_x}
		\end{equation}
		\begin{equation}
		E_{y,out} = \frac{1}{\sqrt{2}} \bigg[  E_0 + i E_{AS,in} \bigg] e^{-i2kL_y}
		\end{equation}
		\end{subequations}
		
		Antisymmetric port output electric field is
		\begin{equation}
		\begin{aligned}
		E_{AS,out} 	&= \frac{1}{\sqrt{2}} (E_{x,out} + iE_{y,out})\\
					&= i e^{-ikL_x-ikL_y} \big[\cos(\Delta \phi) E_0 - \sin(\Delta \phi) E_{AS,in}\big]
		\end{aligned}
		\end{equation}
		where $\Delta \phi = k(L_x-Ly)$
		Output intensity,
		\begin{equation}
		\begin{aligned}
		\mathbf{I}_{AS,out} 		&= \vert E_{AS,out}\vert^2 \\
									&= \bigg[ \cos^2(\Delta \phi)\vert E_0\vert^2 + \sin^2(\Delta\phi)\vert E_{AS,in}\vert^2 - \sin(\Delta\phi)\cos(\Delta\phi) \big[E_0 E^*_{AS,in} + E_0^* E_{AS,in}\big] \bigg]\\
		\Rightarrow	
		\hat{\mathbf{I}}_{AS,out} 	&= \bigg[ \cos^2(\Delta \phi)a_1^{\dagger}a_1 + \sin^2(\Delta\phi)a_2^{\dagger}a_2 - \sin(\Delta\phi)\cos(\Delta \phi) \big[a_1^{\dagger}a_2 + a_2^{\dagger}a_1 \big] \bigg]\\	
		\end{aligned}
		\end{equation}
		
		The expectation value for the intensity using an input coherent laser with $\alpha$ and quantum vacuum input at the antisymmetric port is
		
		\begin{equation}
		\begin{aligned}
		\langle \hat{\mathbf{I}} \rangle 	&= \bra{\alpha,0} \hat{\mathbf{I}}_{AS,out} \ket{\alpha,0}\\
							&= \cos^2(\Delta \phi) \vert \alpha \vert^2
		\end{aligned}
		\end{equation}
		
		which matches the classical description of a Michelson output.
		
		Following the same methods to calculate the radiation pressure, photon number variance is 
		\begin{equation} 
		\begin{aligned}
		\Delta \mathbf{I} 	&= \sqrt{\langle \mathbf{\hat{I}}^2 \rangle  - \langle \mathbf{\hat{I}} \rangle^2)} \\
							&= \vert \alpha \vert \big[ \cos^2(\Delta \phi)\ \big] 
		\end{aligned}
		\end{equation}
		
		\begin{equation}
		I = N \cos^2(\Delta \phi)
		\end{equation}
		
		\begin{equation}
		\frac{\partial I}{\partial \phi} = - 2 \vert \alpha \vert^2\cos(\Delta \phi) \sin(\Delta \phi)
		\end{equation}
		
		\begin{equation}
		\delta \phi = \sqrt{\frac{\hbar \omega}{ P_{in}}} \cot(\Delta \phi)
		\end{equation}
		Here $\delta \phi$ is the microscopic change in phase due to shot noise, whereas $\Delta\phi$ is the DC offset in the arm lengths to begin with. So in general, the shot noise contribution is dependent on the amount of light present at the anti-symmetric port and this will vary depending on what type of read out that is implemented.
		
		\begin{equation}
		\tilde{h}_{SN}(f) = 8.2\times 10^{-22} \; \bigg[ \frac{\text{Strain}}{\sqrt{\text{Hz}}}\bigg]
		\end{equation}
		Unsurprisingly, the variance is proportional to the amount of photons present and the phase difference between the interferometer arms.
		Interestingly, there are a few subtleties associated with measuring the noise, it is assumed here that the interferometer readout is a simple photodetector located at the antisymmetric port.  However, is it possible to measure the light at two different phases (90 degrees apart) and subtract the results.   Also, in section \ref{michelson}, there is a freedom to choose the readout scheme of the interferometer (heterodyne or homodyne) which will also affect the overall quantum noise.

		Losses!?!?!?!?!?!?
		
		\subsubsection{Seismic Noise}
		Seismic noise will be the low frequency barrier for all terrestrial gravitational-wave detectors.  By using seismic isolation platforms and quadruple suspensions, the noise contribution can be attenuated for frequencies larger than ~1Hz.
		
		Compare low and high noise microseisms
		\cite{BlairBook}
		
		\subsubsection{Thermal Noise}
		Thermal Noise [\cite{SaulsonThermalNoise}]
		
		Suspension Thermal Noise \cite{SaulsonThermalSus}
		
		Substrate Thermal Noise [\cite{Saulson}]
		
		Coating Thermal Noise [\cite{HarryThermalCoat}]
		
		Thermal-Optical Noise[\cite{EvansBallmerThermalOptic}]
		
		
		\subsubsection{Newtonian Noise}
		Although there are some noise sources that can be reduced using increasingly complicated techniques such as various levels of seismic isolation or quantum nondemolition devices as shown above. [\cite{SaulsonNewtonian} and \cite{ThorneNewtonian}]
		
		Newtonian noise caused by fluctuating gravitational fields from the wave motion of the ground is one that cannot easily be reduced but possibly be measured and fed-forward or subtracted.[\cite{DriggersNewtonian}]
		
	
	\section{Squeezed States of Light}

	Virtually all undergraduate quantum mechanics textbooks include a section on the harmonic oscillator[Shankar, Sakura].  An interesting result when solving the Schrodinger equation using creation ($\hat{a^{\dagger}}$) and annihilation $\hat{a}$ operators is the existence of a non-zero energy ground state. 
	
	This effect leads to Quantum Noise, which is a fundamental source that can be improved by modifying the quantum vacuum using correlated photons and injecting the states of light into the antisymmetric port of the interferometer.  Within the LIGO community, this procedure of modifying quantum vacuum is called squeezing. Caves analytically derived the effects of quantum noise on the interferometer as well as the improvement due to	
	
	Losses!?!?!?!?!?!?
	References: [Caves, Dwyer, Kwee, Miao]
	