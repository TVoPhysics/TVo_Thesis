%-----------------------------------------------------------------------
%
% File Name: thesis.tex
%
% Author: Kelley, D. B.
%
% Revision: $Id$
%
% Hello and welcome to your new best friend for the next few months! I've created this so that you can have an easier time with formatting and 
% can thus spend more time on making some beautiful and insightful contributions to physics. There are a few useful comments in this file. 
% You should probably take a look at them so you know what options you have. 
%
% This is provided as-is, with no guarantee of accuracy or support. 
% Good luck, and if you get really stuck, figure out where I am and send me a note. I might just respond.
%
%-----------------------------------------------------------------------

% document class and packages
\documentclass[12pt,notitlepage]{report}
\usepackage{bibunits}
\usepackage{suthesis}
\usepackage{graphicx}
\usepackage{color}
\usepackage{amsmath}
\usepackage{amssymb}
\usepackage{amsfonts}
\usepackage[bookmarksnumbered, bookmarksopen, breaklinks, colorlinks, linkcolor=blue, citecolor=magenta]{hyperref}

%\usepackage{caption}
%\usepackage{rotating}
%\usepackage{tensor}

\pdfoutput=1
\DeclareGraphicsExtensions{.pdf,.png}

\hbadness=10000

% new command definitions
\newcommand{\half}{\frac{1}{2}}
\newcommand{\ospsd}{\ensuremath{S_n\left(\left|f_{k}\right|\right)}}

% journal definitions
\newcommand{\apj}{{\it Astrophysical J.}}
\newcommand{\apjl}{{\it Astrophysical J.}}
\newcommand{\aap}{{\it Astron. and Astrophys.}}
\newcommand{\cmp}{{\it Commun. Math. Phys.}}
\newcommand{\grg}{{\it Gen. Rel. Grav.}}
\newcommand{\cqg}{{\it Class. Quant. Grav.}}
\newcommand{\lr}{{\it Living Reviews in Relativity}}
\newcommand{\mnras}{{\it Mon. Not. Roy. Astr. Soc.}}
\newcommand{\pr}{{\it Phys. Rev.}}
\newcommand{\prl}{{\it Phys. Rev. Lett.}}
\newcommand{\prd}{{\it Phys. Rev. D}}
\newcommand{\pra}{{\it Phys. Rev. A}}
\newcommand{\prsl}{{\it Proc. R. Soc. Lond. A}}
\newcommand{\ptrsl}{{\it Phil. Trans. Roy. Soc. London}}
\newcommand{\rmp}{{\it Rev. Mod. Phys.}}

\newcommand{\tcr}{\textcolor{red}}
\newcommand{\tcb}{\textcolor{blue}}
\newcommand{\tcm}{\textcolor{magenta}}
\newcommand{\tcg}{\textcolor{green}}
\newcommand{\tcp}{\textcolor{purple}}

\begin{document}
	\title{Angular trapping of a mirror using radiation pressure}
	\author{David B. Kelley}
	\majorprof{Stefan Ballmer}
	% For the following line, if you got a masters, list it first. Otherwise, leave the first of the two entries blank, like the line below it.
	\previousdegree{M.S. Syracuse University, Syracuse, NY, 2013}{B.S. Massachusetts Institute of Technology, Cambridge, MA, 2010}
	%\previousdegree{}{B.S. Massachusetts Institute of Technology, Cambridge, MA, 2010}
	\submitdate{December 2015} % I know this says submit date, but SU requires that you put the month that you're going to graduate in.
	\degree{Doctor of Philosophy}
	\program{Physics}
	\copyrightyear{2015}
	\majordept{Physics}
	\havededicationtrue
	\dedication{to\\ my parents and Emma}
	\haveminorfalse
	\copyrighttrue
	\doctoratetrue
	\figurespagetrue
	\tablespagetrue
	\electronicsubmitfalse % if false, makes an approval title page. if true, makes title page with no approval spot. 
	%You need to submit a signed version of the approval title page to the office, then an electronic edition with no approval spot. this is dumb.
	
	\Abstract{
		Alignment control in gravitational-wave detectors is really hard.
		I blame Sidles and Sigg \cite{Sidles06}).
		I present the development, implementation, and measurement of a hare-brained scheme which uses radiation pressure to control a suspended mirror,
		trapping it in the longitudinal degree of freedom and one angular degree of freedom. 
	}
	
	\Acknowledgments{
			I would like to thank my thesis advisor Stefan Ballmer for trusting in me as a scientist, always providing indispensable advice, and allowing me to follow my interests.
	
	To Peter Saulson: Your love of science truly inspires me to be a better scientist and teacher.
	
	To Duncan Brown:  Tough questions force us to grow and your group meetings provided a growth spurt.
	
	To Sheila Dwyer, Jenne Driggers, and Georgia Mansell: Thank you for your patience in teaching me the gritty details about the Hanford interferometer, I will never forget the in-chamber work at HAM6 and the long commissioning hours in the control room.
	
	To Daniel Sigg: Your dinners are the perfect fugue of flavors, conversations and friends. Thank you for letting me participate in some of the crucial interferometer commissioning tasks.
	
	To Keita Kawabe: Thank you for the tough questions but also the encouragement to do the best science.
	
	To Rick Savage: I was honored to be a Fellow for so long, thank you for the advice about all aspects of life and the fishing trips when I needed it most.
	
	To Aidan Brooks: Thank you for letting me work on/use TCS at Hanford, your guidance and experience is invaluable in times of confusion.
	
	To Daniel Vander-Hyde: Thank you for being patient with me when things weren't going well but also celebrating with me when things were!
	
	To Dan Brown: Cheers for coming across the pond to help when we needed it most and the Inbetweeners jokes on wings night.
	
	To Corey Gray, Gunner and Gomez: The hikes brought peace to my troubled mind and patience to my restless soul.
	
	To TJ Shaffer: Thank you for helping with steady hands or Guardian skills.
	
	To all the operators and staff at Hanford: Jeff, Betsy, Chandra, TJ, Jim, Cheryl, Nico, Patrick, Ed, Gerardo, Evan.
	
	To Hang Yu: Thank you for helping me understand the inner workings of the interferometer with only a few lines of math and a drawing.
	
	To my forever teachers Melissa Brainerd, Paul Didier, Chrissy Dahms, Carolyn West, and Dan Dorsey:  My work is a direct ripple from your teachings and you inspire kids like me from White Center to chase our dreams.
	
	To my childhood friends Quang, Leah, Thang, Sienna, Khanh, Truc, Kay, Tuan, Wanda, James, Matt and Trung:  You knew me when I was just dreaming of being a scientist, now we're here.
	
	To the Bloomingdale family: Thank you for making me feel at home even in the frigid Syracuse winters.
	
	To Jaysin Lord, Laura Nuttall, and Mat Hutchings: Thank you for the laughs and Thursday wings.
	
	To my family Cuong, Nhi, Tony, Theresa, Brittny, Kayla, Chloe: Your support and encouragement mean everything to me.
	
	To my parents: The difficulties of science are modest in comparison to your journey from Vietnam, this work is dedicated to you.
	
	To my darling Tiffany: Your support provides nourishment for my soul and your love gives a guiding light through the darkest of nights.}
	
	\beforepreface
	\prefacesection{Preface}
	The work presented in this thesis stems from my participation in the LIGO
	Scientific Collaboration (LSC). This work does not reflect the
	scientific opinion of the LSC and it was not reviewed by the collaboration.
	
	\vspace*{0.5cm}
	\noindent The theory of optical trapping in two degrees of freedom in chapter \ref{ch:introduction} is based on
	
	\vspace*{0.25cm}
	
	\noindent A.~Perreca~{\it et~al.}, ``Multi-dimensional optical trapping of a mirror,''
	\prd~{\bf 89} (2014) 122002.
	
	\afterpreface
	
	\Chapter{Introduction}
	\label{ch:introduction}
	\input{introduction}
	%\Chapter{Conclusion}
	%\label{ch:conclusion}
	%\input{conclusion}
	
	%\appendix
	%\Chapter{Beam Separation}
	%\label{ap:beamseparation}
	%\input{beamSeparation}
	
	\clearpage
	\bibliographystyle{unsrt}
	\bibliography{references}
	
	\addcontentsline{toc}{chapter}{\numberline {Bibliography}}
	
	\clearpage
	\birthplacedate{Brooklyn, New York \>\>January 1, 1999}
	\collegewherewhen{%
		\>Massachusetts Institute of Technology \>\>2006--2010, \>B.S.\\
		\>\su	\>\>2010--2015, \>Ph.D.}
	
	\newpage
	\null\vskip1in%
	\begin{center}
		{\Large\bf Curriculum Vitae}
	\end{center}
	\vskip 2em
	\begin{tabbing}
		\tabset
		Title of Dissertation\\
		\>Angular trapping of a mirror using radiation pressure
	\end{tabbing}
	\vskip 1em
	
	\begin{startvita}
	\end{startvita}
	
	\renewenvironment{thebibliography}[1]%
	{\begin{list}{\labelenumi\hss}%
			{\usecounter{enumi}\setlength{\labelwidth}{3em}%
				\setlength{\leftmargin}{5em}}}%
		{\end{list}}
	\renewcommand{\bibitem}[1]{\item\label{#1}\relax}%
	\renewcommand{\theenumi}{\arabic{enumi}}%
	\begin{publications}
		\putbib[papers]
	\end{publications}
	
	\begin{honorarysocieties}
		2010 \> Fellowship??\\
	\end{honorarysocieties}
	
	\finishvita
\end{document}