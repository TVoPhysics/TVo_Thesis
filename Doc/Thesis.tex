\documentclass[10pt,a4paper]{book}
\usepackage[utf8]{inputenc}
\usepackage{amsmath}
\usepackage{amsfonts}
\usepackage{amssymb}
\usepackage{graphicx}
\usepackage{xcolor}
\usepackage{listings}

\title{Adaptive Mode Matching in Advanced LIGO}
\date{Spring 2017}
\author{Thomas Vo}





\begin{document}
		\chapter*{Abstract}
		Going to make LIGO the best possible ever.
	
	\maketitle
		\chapter*{Preface}
		The era of gravitational waves astronomy was ushered in by the LIGO (Laser Interferometer Gravitational-Wave Observatory) collaboration with the detection of a binary black hole collision (Detection paper).  The event that shook the foundation of space-time allowed mankind to view the cosmos in a way that had never been done previously. 
		\addcontentsline{toc}{chapter}{Preface}
	\tableofcontents

\chapter{Introduction}
	\section{Gravitational Waves}
	In 1915, Albert Einstein published his theory of general relativity.
	
	The seminal equation in this theory is:
	
	Soon after 1918, Einstein developed a linear theory of gravity that allowed the creation of gravitational waves.  There was a lot of controversy about whether gravitational waves arise from a physical effect or due to an arbitrary gauge transformation.  Felix Pirani was able to solve the puzzle by attacking the problem in a 
	
	Gravitational waves are the 
	
	Even with the theoretical formulation of gravitational waves resolved, the detection of gravitational waves by ground-based detectors was still a controversial topic among scientists in the field.  This was due to the incredible amount of accuracy needed to actually measure strain. 
	
	Even with some of the most energetic events known to humanity such as the merger of neutron stars and black holes, the amount of strain expected is on the order of 10e-24`
	\section{The LIGO Instrument}
	In the simplest form, the LIGO instrument is an incredibly large Michaelson interferometer.  If we imagine the interferometer as a measure of the differential arm length, it becomes a natural way of detecting gravitational waves.
	
	
	
		
	The LIGO instruments are considered dual-recycled Fabry-Perot interferometers
	
		\subsection{Dual-recycled Fabry-Perot Interferometer}
		A Fabry-Perot cavity is a 
		Power Recycling
		If the interferometer is operating such that the 4 km arms are exactly different in arms a pi over two times the wavelength, then the intensity of the light at antisymmetric port will be close to null.  This means the power from the arms will
		
		Signal Recycling9
		\subsection{Limitations}
		Noise budget:
		- Quantum Noise
		- Seismic
		- Thermal Noise
	\section{Squeezed States of Light}
	The Quantum Noise is a fundamental source that can be helped by squeezing. This is Squeezing (Caves, Dwyer, Kwee, Miao)
		
	\section{The Effects of Mode-Matching}
	Theory section of modematching.
	
	An example of how mode-matching can affect the overall sensitivity.
	
	
	
%%%%%%%%%%%%%%%%%%%%%%	
\chapter{Modeling Mode-Matching}	
	\section{How it works}

	\section{Defining Mode-Matching}
		\subsection{Misalignment}
		Anderson, Kognelik and Li
		\newline
		Guido Paper
		\newline
		
		\subsection{Waist Size and Location}
		Anderson, Kognelik and Li
		\newline
		In contrast to the misalignment orthoganlity 
	
	\section{Finesse Simulations}
		\subsection{ALIGO Design with FC and Squeezer}
		\subsection{Looking at just Modal Change}
		\subsection{QM Limited Sensitivity}
			
	\section{Results}
		* Signal recycling cavity mismatches

		* Mismatches before the OMC
			
		* Mismatch contour graph: Comparing all of ALIGO cavities
			
		* Optical Spring pops up at 7.4 Hz in the Signal-to-Darm TF, re-run with varying SRM Trans which should.

%%%%%%%%%%%%%%%%%%%%%%
\chapter{Mode Matching Cavities at Syracuse}

	\section{Adaptive Mode Matching}
	Real time digital system and model.
	
	\section{Actuators}
		\subsection{Thermal Lenses}
		Fabian's work and UFL paper.
		\subsection{Translation Stages}
		
	\section{Sensors}
		\subsection{Mode Converters}
		
		\subsection{Bullseye Photodiodes}

%%%%%%%%%%%%%%%%%%%%%%
\chapter{Mode Matching Cavities at LIGO Hanford}
	\section{SRC}
	The importance of mode-matching actually goes beyond just reducing the amount of losses in coupled cavities.  It also is important for cavity stability.  If we look at the g-factor of a cavity, it is required through ABCD transformations that the values lay between 0 and 1.  For the signal recycling cavity, if the round trip gouy phase is off by a few millimeters, the stability of the cavity can be compromised. 
	
	\section{Beam Jitter}
	
Current measurements of mode-matching.

%%%%%%%%%%%%%%%%%%%%%%
\chapter{High Power Commissioning}
	\section{Effect on Mode-Matching}
	What is the effect on mode-matching when you change the laser power?

%%%%%%%%%%%%%%%%%%%%%%
\chapter{Solutions for Next Generation Detectors}

* SR3 Heater

* SRM Heater

* Operation: range (in terms of watts and % percentage mismatch)


* Translation stages

* Mechanical description (Solidworks designs)

* Constraints (range, vacuum, alignment, integration)

* Electronics 

* Software
	\listoffigures
	\listoftables


\end{document}