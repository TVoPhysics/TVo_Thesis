\documentclass[oneside]{book}
\usepackage[margin=1.0in]{geometry}
\usepackage{setspace}
\usepackage[utf8]{inputenc}
\usepackage{amsmath}
\usepackage{amsfonts}
\usepackage{amssymb}
\usepackage{graphicx}
\usepackage{xcolor}
\usepackage{listings}
\usepackage{physics}
\usepackage[toc,page]{appendix}
\usepackage{braket}
\usepackage[pagebackref]{hyperref}   


\title{Adaptive Mode Matching in Advanced LIGO}
\date{Spring 2017}
\author{Thomas Vo}


\begin{document}
	
\doublespacing
		\chapter*{Abstract}
		Going to make LIGO the best possible ever.
	
	\maketitle
		\chapter*{Preface}
		The era of gravitational waves astronomy was ushered in by the LIGO (Laser Interferometer Gravitational-Wave Observatory) collaboration with the detection of a binary black hole collision (Detection paper).  The event that shook the foundation of space-time allowed mankind to view the cosmos in a way that had never been done previously. 
		\addcontentsline{toc}{chapter}{Preface}
	\tableofcontents

\chapter{Introduction}

	Say something profound here.

	\section{Gravitational Waves}\label{gravitational waves}
	In 1915, Albert Einstein published his theory of general relativity ref[Einstein].
	
	The seminal equation in this theory is:
	
	\begin{equation} \label{einstein}
	G_{\mu \nu} = 8 \pi T_{\mu \nu}
	\end{equation}
	
	which is a set of 10 coupled second order differential equations that are nonlinear.  In their complete form, equation [1] fully describes the interaction between space-time and mass-energy. To describe the physics in a highly curved space-time, one would have to fully solve the Einstein field equations numerically.  
	
	
	In the weak field approximation, the metric can be described as
	
	\begin{equation} \label{weak}
	g_{\mu \nu}  \approxeq \eta_{\mu \nu} + h_{\mu \nu}
	\end{equation}
	
	where $\eta_{\mu \nu}$ is the metric of flat space time and $|h_{\mu \nu}| \ll 1$ is the perturbation due to a gravitational field.
	
	By plugging in equation \ref{weak} into equation \ref{einstein} and using empty space we obtain the familiar wave equation
	
	\begin{equation} \label{wave}
	\Big(\nabla^2 - \frac{1}{c^2} \pdv[2]{t} \Big) h_{\mu \nu}  = 0
	\end{equation}

	which has a plane-wave solution of the form $h_{\mu \nu} = A_{\mu \nu} e^{ik_{\nu} x^{\nu}}$. 
	
	Using the gauge constraint $h^{\mu \nu}_{,\nu} = 0$, it follows that $A_{\mu \nu} k^{\mu} = 0$ which means that the gravitational wave amplitude is orthogonal to the propagation vector.
	
	Further imposing transverse-traceless gauge and assuming that the wave is traveling in the $x^3$ direction, it can be shown that the complex amplitude has physical significance expressed in the matrix
	
	\begin{equation} \label{gwamp}
	A_{\mu \nu} = 
	\begin{pmatrix}
			0 &    0   &  0      & 0 
		 \\ 0 & A_{xx} &  A_{yx} & 0
		 \\ 0 & A_{xy} & -A_{xx} & 0
		 \\ 0 &    0   &  0      & 0
	\end{pmatrix}
	\end{equation}

	Oftentimes, the four non-zero components of equation \ref{gwamp} can be categorized into two distinct polarizations called plus and cross such that $h_{+} = A_{xx} = -A_{yy}$ and $h_{\cross} = A_{xy} = A_{yx}$ .

 
	It is natural to attempt to understand the physical interpretation of equation \ref{gwamp} as an affect on the position of a free floating particle. Consider the four-velocity, $U^{\alpha}$, in the transverse traceless gauge where the coordinate itself is attached the particles and incorporates any small wiggles that would shake the coordinates.  Of course, any free particles will follow the geodesic equation
	
	\begin{equation}\label{geodesic}
	\nabla U^{\alpha} = \frac{\text{d}}{\text{d} \tau} U^{\alpha} + \Gamma^{\alpha}_{\mu \nu} U^{\mu} U^{\nu} = 0
	\end{equation}	
	where $\Gamma^{\alpha}_{\mu \nu} = \frac{1}{2} g^{\gamma \alpha}(g_{\gamma \mu, \nu} + g_{\gamma \nu,\mu} - g_{\mu \nu, \gamma} )$ are the famous Christoffel symbols.
	By evaluating the first term of the acceleration in equation \ref{geodesic},
	\begin{equation}\label{accel}
	\bigg(\frac{\text{d}U^{\alpha}}{\text{d}\tau}\bigg)_0 = -\Gamma^{\alpha}_{00} 
	\\ = \frac{1}{2} \eta_{\mu \nu} (h_{\beta 0, 0} + h_{0 \beta, 0} + h_{0 0, \beta} )
	\end{equation}
	However, comparing equation \ref{gwamp} and equation \ref{accel}, it is clear that if the particle is initially at rest, then a moment later it is still at rest! The term "at rest" is actually used liberally here since the coordinate system varies along with the gravitational wave. 
	
	Alternatively, one can ask if a gravitational wave passed by a pair of particles separated by length $L$, what would be the effect on the distance between two points?  The proper distance is defined as

	\begin{equation}\label{propdist}
	\delta l
	= \int{g_{\mu \nu} dx^{\mu} dx^{\nu}} \\
	= \int_{0}^{L}{g_{xx} {d}x}\\
	\approx |g_{xx}(x=0)|^{1/2}\\
	\approx [ 1 + \frac{1}{2} h_{xx}(x=0)] L
	\end{equation} 
	
	which shows us two very important points about the nature of gravitational waves.  Firstly, the effect is very small since the length variation is, by definition, a small perturbation in flat space-time.  Secondly, the effect is proportional to the initial separation between the particles. This means a detector which is large will have a better chance to measure these small effects, a point that drove the design of the Laser Interferometer Gravitational-Wave Interferometer (LIGO).
	
	
	\subsection{Energy from a Gravitational Wave}
	
	\subsection{Sources of Gravitational Waves}
	\subsubsection{Compact Binary Inspirals}
	\subsubsection{Continuous Waves}
	\subsubsection{Bursts}
	\subsubsection{Stochastic}
	Even with some of the most energetic events known to humanity such as the merger of neutron stars and black holes, the amount of strain expected is on the order of 10e-24.
	
	\cite{Saulson}
	
	\subsection{Measuring Gravitational Waves with Light}\label{measuringGWs}
	Even with the theoretical formulation of gravitational waves resolved by the 1970s [Pirani], detection of GWs by ground-based detectors was still a controversial topic among scientists in the field.  This was due to the incredible accuracy required to measure the strain from even the most dense astrophysical objects[Rai's paper].  The earliest attempts at detecting these small signals were famously done by Joseph Weber ref[weber] using large resonant bars and piezoelectric transducers to extract the energy from gravitational waves at the resonant frequencies of the bars. Picture of bar detectors[].  However, these bars are limited by thermal noise and can only detect GWs in very narrow frequency bands. 
	
	Interferometers are devices that measure small displacements by using a laser that is split with a partially transmitting mirror (or beamsplitter), which allows 50$\%$ of the light to get reflected and 50$\%$ to be transmitted.  Each of the split beams travel down arms and reflect off of mirrors and return down the arms.  Upon reaching the beamsplitter, the beams recombine and by using the superposition of electromagnetic waves the laser will add linearly at the output port (or antisymmetric port).  The laser beams will gather phase as they propagate down each individual arm, and when recombining, the intensity of the light will be proportional to the phase differences between each beam.  This will correspond to a differential length that is described by
	
	\begin{equation}
	L_{-} = l_{x} - l_{y}
	\end{equation}
	
	As shown in equation \ref{propdist}, the effect of gravitational waves on the proper length between two free falling objects is proportional to the initial separation.  From figure[GWparticles], it is intuitively clear that interferometry would be an ideal technique to detect signals from a gravitational wave.  However, one can explicitly derive a ground-based interferometer's response to a GW from an astrophysical object.
	
	Consider a gravitational wave source arbitrarily located in the sky with respect to an interferometer on Earth. By denoting the interferometer's Cartesian coordinates as $\{\hat{x},\hat{y},\hat{z}\}$ with x and y located along the arms respectively such that the z-axis points directly towards the zenith (i.e. a right-handed system.).  Using the well known Euler angles, a relation from the detector frame to the source frame with coordinates $\{\hat{x}',\hat{y}',\hat{z}'\}$ can be seen in Figure[Euler]. 
	
	If a gravitational wave at the source has emitted GWs with plus and cross polarizations as denoted by equation \ref{gwamp}, then the detector time series can be regarded as ref[Duncan Thesis and AndersonGWs]
	\begin{equation}
	h(t) = F_{+}(\theta,\phi,\psi) \, h_{+}(t) + F_{\times}(\theta,\phi,\psi) \, h_{\cross}(t)
	\end{equation}
	where $F_{+}(\theta,\phi,\psi)$ and $F_{\cross}(\theta,\phi,\psi$ are the antenna pattern functions that project the gravitational wave amplitudes onto the detector frame.
	
	\begin{equation}
	F_{+}(\theta,\phi,\psi) = -\frac{1}{2}[1+\text{cos}^2(\theta)] \text{cos}(2\phi) \text{cos}(2\psi) - \text{cos}(\theta) \text{sin}(2\phi) \text{sin}(2\psi)
	\end{equation}
	\begin{equation}
	F_{\cross}(\theta,\phi,\psi) = + \frac{1}{2}[1+\text{cos}^2(\theta)] \text{sin}(2\phi) \text{cos}(2\psi) - \text{cos}(\theta) \text{sin}(2\phi) \text{sin}(2\psi)
	\end{equation}
	
	If the gravitational wave is located directly above the interferometer (ie $\theta = 0$) and setting $\psi=0$, then the magnitude of the antenna pattern is equal to unity.  Furthermore, by rotating about the $\phi$ angle such that the detector arms align with the plus polarization, the null geodesic equation (ie the path of a photon) in the interferometer frame becomes 
	
	\begin{equation}
	ds^2 = g_{\mu\nu}dx^{\mu} dx^{\nu} = -dt^2 + [1+h_{+}]  dx^2 + [1-h_{+}]  dy^2 + dz^2 = 0
	\end{equation}
	
	Now if the photon is traveling along the x-arm, this means that $dy^2 = dz^2 = 0$ and the metric equation transforms to
	
	\begin{equation}
	\frac{dt}{dx} = \sqrt{ 1+h_{+} } \approx 1+\frac{1}{2} h_{+} 
	\end{equation}
	
	The amount of time required for the photon to reach the x-end mirror (starting at $t=0$) is equal to
	
	\begin{equation}
	t_1 = \int_{0}^{L_{x}} [1+\frac{1}{2}  h_{+}(x) ] \text{d}x
	\end{equation}
	
	where $t_0$ is the start time and $L_x$ is the total length of the x-arm.  Upon returning to the beamsplitter, the photon's total time of flight for the x and y arms are, respectively,
	
	\begin{equation}
	t_2 = 2 L_x + \frac{1}{2} \int_{0}^{L_x} \bigg[  h_{+}(x) +  h_{+}(x + L_x)  \bigg] \text{d}x
	\end{equation}
	\begin{equation}
	t'_{2}= 2 L_y - \frac{1}{2} \int_{0}^{L_y} \bigg[  h_{+}(y) +  h_{+}(y + L_y)  \bigg] \text{d}y
	\end{equation}
	
	If the gravitational wave period is much longer than the time of flight, then $h_{+}$ does not change much during the measurement, which means $h_{+}(\eta_i) \approx h_{+}(\eta_i + L_{\eta_i}) \approx constant$.  By subtracting the flight times of the photons for each arm and setting $L = L_x = L_y$, the difference is proportional to the gravitational wave perturbation multiplied by the sum of arm lengths (with a factor of $c$ to get the units right),
	
	\begin{equation}
	\Delta t = t_2 - t'_{2} = \frac{2L}{c} h_{+}
	\end{equation}
	
	By recasting the expression for time of flight in terms of the phase picked up laser light as it travels through space, the differential phase shift is
	\begin{equation}
	\Delta \Phi = \Phi(t_{2}) - \Phi(t'_{2}) = \frac{4 \pi}{\lambda} \, h_{+} \, L
	\end{equation}
	
	How to actually measure $\Delta \Phi$ with an interferometer is explained in the next section.
	\section{The LIGO Instrument}
	In the simplest form, the LIGO instrument is an incredibly large Michaelson interferometer.  If we imagine the output as a measure of the differential arm length, it becomes a natural way of detecting gravitational waves.
	
	The LIGO instruments are considered dual-recycled Fabry-Perot interferometers.
	
		\subsection{Simple Michaelson}
		As show in figure[michaelson], the interferometer readout uses a photodetector that measure the total power which depends on how laser light in the arms constructively or destructively interferes.  
		At the end of section \ref{measuringGWs}, it was shown that the differential time of flight between photons traveling down each arm carries gravitational wave information.  The difference in flight times, $\Delta t$, can be easily cast into how much phase,$\Delta \phi$ is accumulated by the photons as they propagate through space.  But the question remains how an interferometer actually measures $\Delta \phi$:
		
		If the input electric field of the interferometer is $E_0$, the beamsplitter will transmit $E_0 /2$ down the x-arm and reflect $E_0 /2$ down the y-arm.  By setting the beamsplitter to be the origin,  the beams traveling down their respective arms will have gathered a phase $\phi_i$. Then, upon reflecting off the end mirrors and returning to the beamsplitter, each of the electric fields can be described by these equations
		
			\begin{equation}
			\begin{aligned}
				E_{x} 	&=	\frac{i E_0}{2} e^{2i\phi_{x}}	
			\\	E_{y} 	&=	\frac{i E_0}{2} e^{2i\phi_{y}}
			\end{aligned}
			\end{equation}
			
		Since the electromagnetic waves are linear, the resultant sum of waves at the output will be $E_{out} = E_x + E_y$. A photodiode (PD) is placed at the output (or antisymmetric) port to read out the integrated power which is related to the total electric field by
		\begin{equation}
		\begin{aligned}
			P \vert_{PD}	&= \int_{Area} I \;				\text{d}A 
		\\					&= \int_{Area} (E_x + E_y)^2 \;	\text{d}A 
		\\					&= 
		\end{aligned}
		\end{equation}
		
		In Peter Saulson's book, there is a simple explaination of the light field exiting the anti-symmetric port, but in a more general sense, the phase can be different depending on how the reader chooses to solve Maxwell's equation.
		
		However, the signal has a large DC component which is more difficult to practically detect.  
		
		When working on the dark fringe the signal is proportional to the square of $h(t)$, this is really bad.
		
		So we have to introduce a lock-in detection scheme which uses sidebands to maintain the linear relation between the output and $h(t)$.
		[Black Paper on Signal extraction]
		
		As show in section \ref{gravitational waves}, the gravitational wave will modulate the proper length between two free floating points in space. 
	
		However, the signal has a large DC component which is more difficult to practically detect.  
		
		When working on the dark fringe the signal is proportional to the square of $h(t)$, this is really bad.
		
		So we have to introduce a lock-in detection scheme which uses sidebands to maintain the linear relation between the output and $h(t)$.
		[Black Paper on Signal extraction]	
	
		\subsection{Fabry-Perot Cavities}\label{FP}
		There are two ways to make our instrument better: one is to increase the sensitivity to gravitational waves and the other is to make our noise lower. From equation [], the gravitational wave signal is proportional to the optical path length that the photon travels, which means the most straightforward method of increasing the sensitivity is to make the arms as long as possible (up to the null point described in section []).  Generally, there were two methods to do this: a delay line or a Fabry-Perot resonator.  At the time of writing this thesis, all modern gravitational wave detectors use the latter method.
	
		A Fabry-Perot cavity is an optical system comprised of two mirrors and one laser input. The very simple condition that once the laser has made a round trip around the optics, it is the same shape and size.  Conceptually, this may seem simple but in practice, controlling and sensing any optical cavity comes with many challenges.
		
		Plane wave analysis:
		
		Frequency response of a single FP:
		
		Stability: Ray matrices and resonant condition
		
		In order to prove that the Fabry Perot is stable, it is useful to introduce the matrices that describe a periodic optical system.  A Fabry Perot cavity that is separated by distance $d$ with spherical mirrors that have radii of curvature $R_1$ and $R_2$.
	
		Modal Contents
		
		\subsubsection{Application to LIGO}
		
		
		
		Frequency response of a FP Michaelson:

		References:[Black, Rick's Paper]

		\subsection{Power-Recycled Fabry-Perot Interferometers}
		Power Recycling
		If the interferometer is operating such that the 4 km arms are exactly different in arms a pi over two times the wavelength, then the intensity of the light at antisymmetric port will be close to null.  This means the power from the arms will reflect back towards the input laser.  Ref[] shows the effect of adding a partially reflecting mirror to increase the optical gain of the Michaelson.
		
		
		References: [Meers, Kiwamu]
		
		\subsection{Dual-Recycled Fabry-Perot Interferometers}
		
		DC
		
		Figure comparing the three cases of increased sensitivity.
		
		References: [Kiwamu]
		
		\subsection{Fundamental Noise Sources}
		Ref: Evan Hall, GWINC

		Noise budget:

		\subsubsection{Quantum Noise}
		\subsubsection{Seismic Noise}
		\subsubsection{Thermal Noise}
		\subsubsection{Newtonian Noise}

	
	
	\section{Squeezed States of Light}
	The Quantum Noise is a fundamental source that can be improved by modifying the quantum vacuum using correlated photons and injecting the states of light into the antisymmetric port of the interferometer.  Within the LIGO community, this procedure of modifying quantum vacuum is called squeezing. Caves analytically derived the effects of quantum noise on the interferometer as well as the improvement due to i
	
	References: [Caves, Dwyer, Kwee, Miao]
		
\chapter{The Fundamentals of Mode-Matching}
	Theory section of modematching. Gaussian beams, define relevant quantities, Gouy Phase!
	
	In order to properly explain the effects of electromagnetic waves in the LIGO interferometers, it is useful to review the  
	
	References: [Kognelik and Li]
	
		\section{Gaussian Beam Optics}
		When dealing with length sensing degrees of freedoms such as section \ref{FP}, the simple plane wave approximation is sufficient in describing the dynamics, however, when trying to understand the misalignment and mode-mismatch signals, it is necessary to incorporate Gaussian beams and associated their higher order modes.

		
		Consider the famous Maxwell's equations in vacuum:
		
		\begin{equation}
		\label{18.1:1}
		\begin{aligned}
		 \nabla \times \mathbf{E} &=-\frac{\partial \mathbf{B}} {\partial t},&
		\\\nabla \cdot \mathbf{B} &=0&,
		\\\nabla \times \mathbf{B} &= \mu\ \mathbf{J} + \frac{1}{c^2} \frac{\partial \mathbf{E}} {\partial t}&
		\\
		\nabla \cdot \mathbf{E} &= \frac{\rho}{\epsilon}&
		\end{aligned}
		\end{equation}
		
		
		Concentrating on the electric field in vacuum, we arrive at the Helmholtz Equation
		
		\begin{equation}\label{Helmholtz}
		(\nabla^2 + k^2 ) \mathbf{U}(\mathbf{r},t) = 0
		\end{equation}
	
		
		where $k=\frac{2\pi\nu}{c}$ is the wave number and $\mathbf{U}(\mathbf{r},t)$ is the complex amplitude which can describe either the electric or magnetic fields.  
		
		It is possible to express the solution to equation  \ref{Helmholtz} as
		a plane wave with a modulated complex envelope
		\begin{equation}
		\mathbf{U}(\mathbf{r}) = \mathbf{A}(\mathbf{r}) e^{-ikz)}
		\end{equation}
		
		By imposing the constraints which force the envelope to vary slowly with respect to the z-axis within the distance of one wavelength $\lambda = 2\pi/k$,

		\begin{subequations}
		\begin{equation}\label{paraxiala}
		\bigg| { \partial^2 \mathbf{A} \over \partial z^2 } \bigg|  \ll  \bigg| { k {\partial \mathbf{A} \over \partial z} } \bigg|
		\end{equation}
		\begin{equation}\label{paraxialb}
		\bigg| { \partial^2 \mathbf{A} \over \partial z^2 } \bigg|  \ll  \bigg| { k {\partial^2 \mathbf{A} \over \partial x^2} } \bigg|
		\end{equation}
		\begin{equation}\label{paraxialc}
		\bigg| { \partial^2 \mathbf{A} \over \partial z^2 } \bigg|  \ll  \bigg| { k {\partial^2 \mathbf{A} \over \partial y^2} } \bigg|
		\end{equation}
		\end{subequations}
		
		the partial differential equation which arises is called the Paraxial Helmholtz Equation:
		
		\begin{equation}\label{paraHelmholtz}
		\nabla_T^2 A(r) - i2k {\partial A(r) \over \partial z} = 0
		\end{equation}
		
		where $\nabla_T^2 = {\partial^2  \over \partial x^2} + {\partial^2  \over \partial y^2} $ is the transverse Laplacian.  A simple solution for equation \ref{paraHelmholtz} is the complex paraboloidal wave
		
		\begin{equation} \label{complexenvelope}
		A(\mathbf{r}) = \frac{A_0}{q(z)} e^{\frac{-ikr^2}{2q(z)}} , \quad q(z)=z+iz_0
		\end{equation}
		
		where $z_0$ is the Rayleigh range and is directly proportional to the square of the waist size.  In order to separate the amplitude and phase portions of the wave, it is useful to rewrite $q(z)$ as
		
		\begin{equation}\label{invq}
		\frac{1}{q(z)} = \frac{1}{R(z)} - i \frac{\lambda}{\pi W^2(z)}
		\end{equation} 
		
		Plugging equation \ref{invq} into \ref{complexenvelope} leads directly to the complex amplitude for a Gaussian Beam
		
		\begin{equation}
		U(r) = A_0 \frac{W_0}{W(z)} e^{-\frac{r^2}{W^2(z)}} e^{-ikz - ik \frac{r^2}{2R(z)} + i \phi(z)}
		\end{equation}
		
		where
		\begin{subequations}
		\begin{equation}
		W(z) = W_0 \sqrt{1 + \bigg( \frac{z}{z_0} \bigg)^2}
		\end{equation}
		\begin{equation}\label{ROC}
		R(z) = z \bigg[ 1 + \bigg( \frac{z}{z_0} \bigg)^2 \bigg]
		\end{equation}
		\begin{equation}
		\phi(z)= \text{tan}^{-1}\bigg(\frac{z}{z_0}\bigg)
		\end{equation}
		\begin{equation}
		W_0 = \sqrt{\frac{\lambda z_0}{\pi}}
		\end{equation}
		\end{subequations}


		\subsubsection{Gouy Phase}
		The Gouy phase is 
		- Gouy Phase + graph
		\subsubsection{Intensity}
		- Intensity and Power
		
		\subsubsection{Hermite-Gauss Modes}
		The fundamental Gaussian beam is not the only solution which can be used to solve equation \ref{paraHelmholtz}.  In fact, there exists a complete set of solutions that can solve the paraxial Helmholtz Equation, which are referred to as the higher order modes (HOMs)
		\begin{equation}\label{HG}
		\begin{aligned}
		&U_{mn}(x,y,z) = A_{mn}\bigg[ \frac{W_0}{W(z)} \bigg] \mathbb{G}_m\Bigg( \frac{\sqrt{2}x}{W(z)}  \Bigg) \mathbb{G}_n\Bigg( \frac{\sqrt{2}y}{W(z)} \Bigg)\\
		&\times \text{exp} \bigg\{ -ikz - \frac{ik(x^2+y^2)}{2R(z)} + i(m+n+1)\phi(z) \bigg\}
		\end{aligned}
		\end{equation}
		
		where,
		\begin{equation}
		\mathbb{G}_(u) = \mathbb{H}(u) \, \text{exp}(-u^2/2)
		\end{equation}
		
		and $ \mathbb{H}(u)$ are the well known Hermite polynomials.  It is important to mention that the Gouy phase of the complex amplitude is different than the fundamental Gaussian beam by a factor of $(m + n + 1)$. Also, the intensity distribution of these higher order modes are much different. Both of these facts will become extremely important in the following wavefront sensing discussion.
		
		It is useful to normalize the Hermite-Gauss modes with respect to the overall power, which are derived in Appendix[].
		
		\subsubsection{Laguerre Modes}
		Another complete set of alternative solutions to equation \ref{paraHelmholtz} exists which are called the Laguerre-Gauss modes
		
		\begin{equation}\label{LG}
		\begin{aligned}
		&V_{\mu\nu}(\rho,\theta,z) = A_{\mu\nu}\bigg[ \frac{W_0}{W(z)} \bigg] \mathbb{L}^{\mu}_{\nu} \Bigg( \frac{\sqrt{2}x}{W(z)}  \Bigg) \\
		&\times \text{exp} \bigg\{-ikz-\frac{ik\rho^2}{2R(z)} + i(\mu+2\nu+1)\phi(z) \bigg\}
		\end{aligned}
		\end{equation}
		
		where $\mathbb{L}^{\mu}_{\nu} \Bigg( \frac{\sqrt{2}x}{W(z)}  \Bigg)$ is the Laguerre polynomial function. Both equations \ref{HG} and \ref{LG} are able to fully describe any complex electromagnetic amplitude; and because they both form complete sets, there is a rotation which can map from one basis to the other [ref Bond and Biejergasern]
		
		\begin{equation}
		U^{LG}_{\mu \nu} (x,y,z) = \sum\limits_{k}^{N} i^k b(n,m,k) U^{HG}_{N-k,k} (x,y,z)
		\end{equation}
		where
		\begin{equation}
		b(n,m,k) = \sqrt{\bigg( \frac{(N-k)!k!}{2^N n!m!} \bigg)} \frac{1}{k!} \frac{\text{d}^k}{\text{d}t^k}[(1-t)^m (1+t)^m]\vert_{t=0}
		\end{equation}

		\subsection{Misalignment and Higher Order Modes}\label{Misalign}
		Morrison and Anderson derived a simplistic way of how small misalignments and mismodematched cavities can couple the fundamental Gaussian beam into various higher order modes.  This is done by taking a linear cavity and using its perfectly matched Gaussian beam as a reference, and then varying the input electric field with small perturbations and expanding in terms of the cavity modes.  As long as the mismatches are small, it is possible to consider only the first few terms of the expansion which have gained power from the fundamental mode.
		
		Consider the first three modes of equation \ref{HG} in one dimension and normalized to set the total optical power to unity (derived in Appendix[]):

		\begin{equation}
		\label{Gauss1D}
		\begin{aligned}
				U_{0}(r) & =	\bigg( \frac{2}{\pi w^2(z)} \bigg)^{1/4}  e^{-r^2/w^2(z)}		&
		\\		U_{1}(r) &	=	\bigg( \frac{2}{\pi w^2(z)} \bigg)^{1/4}  \frac{2r}{w(z)} \quad e^{-r^2/w^2(z)}&,
		\\	 	U_{2}(r) &	=	\bigg( \frac{2}{\pi w^2(z)} \bigg)^{1/4}  \frac{1}{\sqrt{2}} \bigg( \frac{4r^2}{w^2(z)} - 1 \bigg)   e^{-r^2/w^2(z)}
		\end{aligned}
		\end{equation}
				
		
		\subsubsection{Beam Axis Tilted}
		If the input beam into an optical cavity is tilted by an angle $\alpha$ with respect to the nominal cavity axis as shown in Figure [], the wave front of the input beam will have an extra phase propagation relative to the cavity that is approximately proportional to $e^{ik \alpha r}$.  By implementing the small angle approximation, which is valid if the misalignment is much smaller than the divergence angle of the fundamental mode $k \alpha r << 1$, the resultant input beam is
		
		\begin{equation}
		\Psi \approx U_{0}(r) e^{ik \alpha r} \approx U_{0}(r) ( 1 + ik \alpha r ) =  U_{0}(r) + \frac{ik \alpha w(z)}{\sqrt{2\pi}} U_{1}(r)
		\end{equation}
		
		Here the factor associated with the first higher order mode is complex, indicating there is a 90 degree phase difference between the fundamental and off axis mode. 
		\subsubsection{Beam Axis Displaced}
		If the input beam is displaced in the transverse direction by a quantity $\Delta r$, the resultant waveform will be
		
		\begin{equation}
		\begin{aligned}
			\Psi 	&=  		U_{0}(r + \Delta r	) 
			\\		&= 			\bigg( \frac{2}{\pi w^2(z)} \bigg)^{1/4}  e^{-(r+\Delta r)^2/w^2(z)}
			\\		&= 			\bigg( \frac{2}{\pi w^2(z)} \bigg)^{1/4}  e^{-(r^2 + 2r \Delta r  + \Delta r^2)/w^2(z)}
			\\		&\approx 	\bigg( \frac{2}{\pi w^2(z)} \bigg)^{1/4}  e^{-r^2/w^2(z)} e^{-2r \Delta r/w^2(z)}
			\\		&\approx 	\bigg( \frac{2}{\pi w^2(z)} \bigg)^{1/4}  e^{-r^2/w^2(z)} \bigg(1 - \frac{2r \Delta r}{w^2(z)} \bigg)
			\\		&=			\bigg( U_0(r) - \sqrt{\frac{2}{\pi}} \frac{\Delta r }{w(z)} U_1(r)	 \bigg)
		\end{aligned}
		\end{equation} 
		
		Similarly to a tilted input beam axis, the displaced beam axis couples power to the first higher order mode, however, the latter does not have a 90 degree phase difference previously seen in the former.  This point is of extreme importance when trying to discern between the two effects as shown in Section [].  Although comparing the two cases in Figure, one can already seen the difference between the wavefronts in the near field, $z<<z_R$, and the far field, $z>>z_R$.  
		
		In the near field, there is no phase difference due to a displaced beam, but there is one for a tilted beam.  Conversely, in the far field, there is no phase difference due to a tilted beam, but there is one from a displaced beam.  In order to implement a closed loop feedback system, the wavefront sensors discussed in Section \ref{WFS} will use this precise logic to extract an error signal.
		
		
		\subsection{Mode Mismatch and Higher Order Modes}\label{Modemismatch}
		
		
		\subsubsection{Waist Size Shifted}
		By considering the effect of evaluating the fundamental mode at the waist position, $z=0$, but changing the waist size by a small amount $\epsilon$, it is possible to see coupling into higher order modes by expanding to first order.	
		\begin{equation}
		\begin{aligned}
		\Psi 	&=  		U_{0} \big(r,w(z) = w_0/(1+\epsilon) \big) 
		\\		&= 			\bigg( \frac{2}{\pi w_0^2} \bigg)^{1/4} \sqrt{1 + \epsilon} \quad e^{-r^2 (1+\epsilon)^2/w_0^2 }
		\\		&\approx 	\bigg( \frac{2}{\pi w_0^2} \bigg)^{1/4} (1 + \epsilon /2) \quad e^{-r^2/w_0^2} \quad e^{-2r^2\epsilon/w_0^2} 
		\\		&\approx 	\bigg( \frac{2}{\pi w_0^2} \bigg)^{1/4} (1 + \epsilon /2) \quad e^{-r^2/w_0^2} \quad (1-2r^2\epsilon/w_0^2)
		\\		&\approx 	\bigg( \frac{2}{\pi w_0^2} \bigg)^{1/4} \bigg(1+ 2\epsilon\bigg(\frac{1}{4} - \frac{r^2}{w_0^2}\bigg) \bigg ) \quad e^{-r^2/w_0^2}	
		\\		&=			U_0(r) \, - \, \frac{\epsilon}{\sqrt{2}} \, U_2(r)
		\end{aligned}
		\end{equation}
		Changing the waist size by a small amount will couple the fundamental mode to the in-phase second order Hermite Gauss mode.
		
		\subsubsection{Waist Position Shifted}
		To repeat the process from above with a waist position shift, it is useful to start with a more general equation that includes the phase that is gained from including the radius of curvature,
		
		\begin{equation}\label{EFieldwPhase}
		\Psi = 	\bigg( \frac{2}{\pi w^2(z)} \bigg)^{1/4} \quad e^{-r^2/w^2(z)} \quad e^{-ikr^2/2R(z)}
		\end{equation}
		where $R(z)$ is from equation \ref{ROC}.  It is also useful to approximate the shift in waist position along the longitudinal axis is small compared to the Rayleigh range of the beam, $\Delta z << z_0$, which leaves the waist size approximately the same and the radius of curvature inversely proportional to the shift. 
		
		\begin{subequations}
		\begin{align}
		\begin{split}
			w^2(\Delta z)	&= 	w^2_0 \bigg[1 + \bigg(\frac{\Delta z}{z_0}  \bigg)^2 \bigg]  \approx	w^2_0
		\end{split}\\
		\begin{split}
			R(\Delta z) 	&=	\Delta z \bigg(1 + \bigg(\frac{z_0}{\Delta z}\bigg)^2\bigg) \approx	\frac{z_0^2}{\Delta z}
		\end{split}
		\end{align}
		\end{subequations}

		Plugging the equations above into \ref{EFieldwPhase},
		
		\begin{equation}
		\begin{aligned}
		\Psi 	&\approx	\bigg( \frac{2}{\pi w_0^2} \bigg)^{1/4} \quad e^{-r^2/w_0^2} \quad e^{-ikr^2 \Delta z /2 z_0^2}
		\\		&\approx	\bigg( \frac{2}{\pi w_0^2} \bigg)^{1/4} \quad e^{-r^2/w_0^2} \quad \bigg( 1-\frac{ikr^2 \Delta z}{2 z_0^2}  \bigg)
		\\		&=			U_0(r) - \bigg( \frac{2}{\pi w_0^2} \bigg)^{1/4} e^{-r^2/w_0^2} \quad \frac{ikr^2 \Delta z}{2 z_0^2}
		\\		&=			U_0(r) - i \frac{\Delta z}{2k w_0^2} \bigg( 4U_2(r) + U_0(r) \bigg)
		\end{aligned}
		\end{equation}
		
		The equations above show that a fundamental Gaussian mode that is shifted in waist position will couple power to the second order Hermite Gauss mode.  Although changes in the waist size or position couple power to the same mode, they differ by a 90 degrees in phase as denoted by the extra factor of $i$ in the coupling coefficient.  By recognizing the two effects are in different quadrature phases will allow a user to design a system to distinguish between the different types of physical couplings, this is shown in Section \ref{WFS}.
		
		 In order to be physically valid one would need to consider the full two dimensional space so that the equation would encapsulate the full transverse mode, however, the x and y components would follow the exact same derivation. On that point, it is important to note that only the mode mismatch couplings from either a varying waist position or size has higher order modes that are circularly symmetric.
		
		
		
		
%%%%%%%%%%%%%%%%
		\section{Wavefront Sensing}\label{WFS}
		Heterodyne detection via modal decomposition of the full electric field allows the use of wavefront sensors to extract an error signal from the optical system.  Hefetz et.al Ref[Sigg and Nergis] created a formalism to describe the use of wavefront sensors by creating frequency sidebands which accumulate a different Gouy phase than the electric field at the carrier frequency when passed through the optical system.  By observing the demodulated signal of the intensity, it is possible to obtain a linear signal that corresponds to a physical misalignment or mode mismatch.
		
		Fundamentally, the purpose of wavefront sensing is to detect the content of higher order modes due to physical disturbances of the optical cavity (ie. mode mismatch or misalignment).  In other words, it is examining the difference of basis sets between the incoming eigenmodes and the cavity eigenmodes.
		
		Consider a general equation for an electric field which is a linear combination of all higher order modes of the complex amplitude
		\begin{equation}
		E(x,y,z) = \sum\limits_{m,n}^{\infty} a_{mn} U_{mn}(x,y,z)
		\end{equation}
		
		where $ U_{mn}(x,y,z)$ are the eigenmodes described in equation \ref{HG} (or \ref{LG}) and $a_{mn}$ is the complex amplitude.  It is also convenient in the following analysis to use vectors when describing the composition of the electric fields.
		
		\begin{equation}
		\ket{E(x,y,z)} = \begin{pmatrix} E_{00} 
		\\ E_{01}
		\\E_{10}
		\\E_{20}
		\\E_{02}
		\end{pmatrix}
		\end{equation}

		When creating a theory that involves laser beams, it is useful to define operators that are important in describing physical situations.  For example, laser beams propagate through space and pick up phase according to equation \ref{HG} which can be represented by the spatial propagation operator,
		
		\begin{equation}
		\hat{P}_{mn,kl} = \delta_{mn} \delta_{kl} \quad \text{exp}[-ik(z_2 - z_1)] 
		\\ \text{exp}[i(m+n+1)\phi(z)]
		\end{equation}

		However, it is useful to compare how the fundamental Gaussian mode propagates compared to the higher order modes,

		\begin{equation} \label{GouyPhaseMatrix}
		\hat{\eta}_{\mu \nu} = 
		\begin{pmatrix}
		e^{i\phi}	&0			&0			& 0 			& 0
		\\ 0		&e^{2i\phi}	&0			& 0				& 0
		\\ 0		&0			&e^{2i\phi}	& 0				& 0
		\\ 0		&0			&0			& e^{3i\phi}	& 0
		\\ 0		&0			&0			& 0				& e^{3i\phi}
		\end{pmatrix}
		\end{equation}

		From the above diagonal elements, it is clear that the higher order modes have an extra phase compared to the fundamental 00 mode, this effect will be extremely important on how an error signal can be derived from the optical system.
		
		\begin{equation}
		\ket{E(x,y,z_2)} = \hat{M}(x,y,z_1,z_2)	\ket{E(x,y,z_1)}
		\end{equation}
		
		where $\hat{M}(x,y,z_1,z_2)$ is the misalignment operator.  Since we are using the paraxial approximation, the z-components of the misalignment operator are small so we can approximate $\hat{M}_(x,y,z_1,z_2) \approx \hat{M}_(x,y)$ and the expectation value is
		
		\begin{equation}
		M_{mn,kl}=  \bra{U_{mn}(x,y,z_1)} M(x,y) \ket{U_{kl}(x,y,z_2)}
		\end{equation}

		where the product is an integral over the transverse space $\int \!\!\! \int_{D(x,y)} \text{d}x \text{d}y$
		

		\begin{equation} \label{misalign_matrix}
		\hat{\Theta}_{\mu \nu} = 
		\begin{pmatrix}
		   1			&2i\theta_x		&2i\theta_y		& 0 & 0
		\\ 2i\theta_x	&1				&0				& 0	& 0
		\\ 2i\theta_y	&0				&1				& 0	& 0
		\\ 0			&0				&0				& 1	& 0
		\\ 0			&0				&0				& 0	& 1
		\end{pmatrix}
		\end{equation}

		\begin{equation} \label{mistrans_matrix}
		\hat{\mathbb{D}}_{\mu \nu} = 
		\begin{pmatrix}
			1					&\alpha_x/\omega_{0}	&\alpha_y/\omega_{0}	& 0 & 0
		\\ \alpha_x/\omega_{0}	&1						&0						& 0 & 0
		\\ \alpha_y/\omega_{0}	&0						&1						& 0	& 0
		\\ 0					&0						&0						& 1	& 0
		\\ 0					&0						&0						& 0 & 1 
		\end{pmatrix}
		\end{equation}
		
		
		\begin{equation} \label{waistloc_matrix}
		\hat{\mathbb{Z}}_{\mu \nu} = 
		\begin{pmatrix}
		1				&0		&0		&\Delta z_x 	&\Delta z_y  
		\\ 0			&1		&0		&0 				&0
		\\ 0			&0		&1		&0 				&0
		\\ \Delta z_x	&0		&0		&1 				&0
		\\ \Delta z_y	&0		&0		&0				&1 
		\end{pmatrix}
		\end{equation}
		
		where $\Delta z_{(x,y)} =  \frac{i}{\sqrt{2}} \frac{\lambda b}{2\pi\omega_{0}} $
		
		
		\begin{equation} \label{waistsize_matrix}
		\hat{\mathbb{Z}}_{0, \mu \nu} = 
		\begin{pmatrix}
		1					&0		&0		&\Delta z_{0,x} 	&\Delta z_{0,y} 
		\\ 0				&1		&0		&0 					&0
		\\ 0				&0		&1		&0 					&0
		\\ \Delta z_{0,x} 	&0		&0		&1 					&0
		\\ \Delta z_{0,y} 	&0		&0		&0					&1 
		\end{pmatrix}
		\end{equation}

		where $\Delta z_{0,(x,y)} =   \frac{1}{\sqrt{2}} \frac{\omega'-\omega_{0}}{\omega_{0}} $

		Figure: Beat frequency between two modes.
		
		
		\subsubsection{Example: Fabry Perot Cavity}
		
		
		\subsubsection{Example: Simple Michaelson}
		


		\section{Effects of Mode-matching on Squeezing}
		Still not clear to me.
		In the most simple sense, a loss is akin to coupling quantum vacuum into a squeezed state.
		References: [Miao, Sheon, Kimble, Dwyer]
	
	
%%%%%%%%%%%%%%%%%%%%%%	
\chapter{Simulating Mode-Matching with Finesse}	
	\section{How it works}
	Summary of how Finesse works (input output matrix), how it handles HOMs
	\section{Finesse Simulations}
		\subsection{ALIGO Design with FC and Squeezer}
		\subsection{Looking at just Modal Change}
		\subsection{QM Limited Sensitivity}
			
	\section{Results}
		* Signal recycling cavity mismatches

		* Mismatches before the OMC
			
		* Mismatch contour graph: Comparing all of ALIGO cavities
		
		* Optical Spring pops up at 7.4 Hz in the Signal-to-Darm TF, re-run with varying SRM Trans which should.

%%%%%%%%%%%%%%%%%%%%%%
\chapter{Experimental Mode Matching Cavities at Syracuse}
In conjunction with Sandoval et al, the adaptive modematching table top experiment was able to show the feasibility of a fully dynamical system.
	\section{Adaptive Mode Matching}
	Real time digital system and model.
	
	Signal chain.
	
	\section{Actuators}
		\subsection{Thermal Lenses}
		Fabian's work and UFL paper.
		\subsection{Translation Stages}
		
	\section{Sensors}
		\subsection{Bullseye Photodiodes}
		As stated in section [], we saw that the error signal for a mode-mismatched cavity has cylindrical symmetry due to the beating between the LG-01 and the LG-00 mode.  This means that the quadrant photodiodes would not have a way to detect the modal content of a cavity.  One way to solve this is to introduce a type of detector that can sense the power outside versus the power inside (ugh re write this).
		
		In [Rana's IO final design document] 
		
		
		Why do we need bullseye photodiodes, meantion the geometry of the error signal.
		
		Derivations in the appendix.
		
		Picture of BPD
		
		Pitch and Yaw sensing matrix
		
		Explain why we need $\omega_{0} = \sqrt{2} r_0$
		
		The ratio of the out over in will give:
		
		\begin{equation}
		\text{Power Ratio} = \frac{\text{P}_2 + \text{P}_3 + \text{P}_4}{\text{P}_1}  \\
		= \frac{e^{-2r_0^2/ \omega_{0}^2}} {1 - e^{-2r_0^2/ \omega_{0}^2 }} \approx 0.582
		\end{equation}
		
		\subsection{Mode Converters}
		
		\subsection{Scanning Gaussian Beams}
		

%%%%%%%%%%%%%%%%%%%%%%
\chapter{Mode Matching Cavities at LIGO Hanford}

	\section{Active Wavefront Control System}
	
	\section{SRC}
	The importance of mode-matching actually goes beyond reducing the amount of losses in coupled cavities.  It also is important for cavity stability.  If we look at the g-factor of a cavity, it is required through ABCD transformations that the values lay between 0 and 1.  For the signal recycling cavity, if the round trip gouy phase is off by a few millimeters, the stability of the cavity can be compromised. 
	
	
	\section{Beam Jitter}
	
	Current measurements of mode-matching.
	
	\section{Contrast Defect}

%%%%%%%%%%%%%%%%%%%%%%

\chapter{Solutions for Detector Upgrades} 

* A full modal picture, sensors and actuators

* SR3 Heater

* SRM Heater

* Bullseye photodetectors

* Operation: range (in terms of watts and % percentage mismatch)

* Translation stages

* Mechanical description (Solidworks designs)

* Constraints (range, vacuum, alignment, integration)

* Electronics 

* Software


	\listoffigures
	\listoftables

\begin{appendices}

	\chapter{Resonator Formulas}
	
	
	Ray Trace: Round trip phase, cavity stability
	
	TF: analytical derivation
	
	Important quantities: Finesse, cavity pole, free spectral range
	
	Effect of higher order modes into the cavity, mode scanning. All comes from round trip Gouy phase.
	
	
	\chapter{Hermite Gauss Normalization}
	According to equation \ref{HG}, the higher order modes in the Hermite Gauss basis has the intensity profile,
	
	\begin{equation}
		I_{mn} (x,y,z) = \vert A_{mn} \vert^2 \bigg[ \frac{W_0}{W(z)} \bigg]^2  \mathbb{G}^2_n\Bigg( \frac{\sqrt{2}x}{W(z)} \Bigg) \mathbb{G}^2_n\Bigg( \frac{\sqrt{2}y}{W(z)} \Bigg)
	\end{equation}

	It is useful to normalize the first few lowest order modes with respect to the total optical power since the Gaussian beam will couple to them the most due either misalignment or mode mismatch as seen in section [].
	
	In one dimension, the total optical power for the first 3 modes are
	
	\begin{equation}
	\label{HGNormalInt1D}
	\begin{aligned}
		P_{0}(x,y,z) 	& 	=	\int_{-\infty}^{\infty}  \vert A_{0} \vert^2   \bigg[ \frac{W_0}{W(z)} \bigg] e^{-2x^2/w^2(z)} dx	&
	\\	P_{1}(x,y,z)	&	=	\int_{-\infty}^{\infty}  \vert A_{1} \vert^2  \bigg[ \frac{W_0}{W(z)} \bigg] \frac{8x^2}{w^2(z)} 	
								e^{-2x^2/w^2(z)}dx &
	\\	P_{2}(x,y,z)	&	= 	\int_{-\infty}^{\infty}  \vert A_{2} \vert^2   \bigg[ \frac{W_0}{W(z)} \bigg] \bigg(\frac{8x^2}{w^2(z)}	-2\bigg)^2e^{-2x^2/w^2(z)}dx
	\end{aligned}
	\end{equation}
	
	In two dimensions, the total optical power for the first 3 modes are
	\begin{equation}
	\label{HGNormalInt2D}
	\begin{aligned}
		P_{00}(x,y,z) 	& 	=	 \int_{-\infty}^{\infty} \int_{-\infty}^{\infty}  \vert A_{00} \vert^2   \bigg[ \frac{W_0}{W(z)} \bigg]^2 e^{-2x^2/w^2(z)}e^{-2y^2/w^2(z)} dx dy&
	\\	P_{10}(x,y,z)	&	=	\int_{-\infty}^{\infty} \int_{-\infty}^{\infty}  \vert A_{10} \vert^2  \bigg[ \frac{W_0}{W(z)} \bigg]^2 \frac{8x}{w^2(z)} e^{-2x^2/w^2(z)}e^{-2y^2/w^2(z)} dx dy&
	\\	P_{20}(x,y,z)	&	= 	\int_{-\infty}^{\infty} \int_{-\infty}^{\infty}  \vert A_{20} \vert^2   \bigg[ \frac{W_0}{W(z)} \bigg]^2 \bigg(\frac{8x^2}{w^2(z)} - 2\bigg)^2 e^{-2x^2/w^2(z)}e^{-2y^2/w^2(z)} dx dy
	\end{aligned}
	\end{equation}
	
	By setting the equations above to unity, the normalization factors become
	
	\begin{equation}
	\begin{aligned}
		A_{0} &	= \bigg( \frac{2}{\pi w_0^2} \bigg)^{1/4} 
	\\	A_{1} &	= \bigg( \frac{2}{\pi w_0^2} \bigg)^{1/4} \frac{1}{\sqrt{2}}
	\\	A_{2} &	= \bigg( \frac{2}{\pi w_0^2} \bigg)^{1/4} \frac{1}{\sqrt{8}}
	\end{aligned}
	\end{equation}
	
	\begin{equation}
	\begin{aligned}
		A_{00} &	= \sqrt{\frac{2}{\pi w_0^2}}
	\\	A_{10} &	= \sqrt{\frac{1}{\pi w_0^2}}
	\\	A_{20} &	= \sqrt{\frac{1}{4\pi w_0^2}}
	\end{aligned}
	\end{equation}
	
	Therefore the normalized modes are
	
	
	
	


	
	

	
	
	
	\chapter{Bullseye Photodiode Characterization}
	
	\section{DC}
	\begin{equation}
	\begin{split}
	\text{Power} &= \int_{A}^{B} \abs{A_{00}}^2e^{\frac{-2r^2}{\omega_{0}^2}} 2\pi r dr\\
			&= -\abs{A_{00}}^2 \frac{\pi \omega_{0}^2}{2} e^{\frac{-2r^2}{\omega_{0}^2}} \biggr\rvert_A^B
	\end{split}
	\end{equation}
	
	\begin{equation}
	\text{P}_{in} = \text{Power} \biggr\rvert_0^{r_0} = \abs{A_{00}}^2 \frac{\pi \omega_{0}^2}{2} [1 - e^{\frac{-2r_0^2}{\omega_{0}^2}}]
	\end{equation}
	
	\begin{equation}
	\text{P}_{out} = \text{Power} \biggr\rvert_{r_0}^{\infty} = \abs{A_{00}}^2 \frac{\pi \omega_{0}^2}{2} [e^{\frac{-2r_0^2}{\omega_{0}^2}}]
	\end{equation}
	
	\begin{equation}
	\text{P}_{total} =  \text{P}_{in} + \text{P}_{out}
	\end{equation}
	
	\begin{equation}
	\omega = \sqrt{\frac{\text{P}_{total}}{\abs{A_{00}}^2 \pi / 2}}
	\end{equation}
	
	\begin{equation}
	\text{DC Power Ratio} 
	= \frac{P_{out}}{P_{in}} \\
	= \frac{e^{-2r_0^2/ \omega_{0}^2}} {1 - e^{-2r_0^2/ \omega_{0}^2 }} \approx 0.582
	\end{equation}
	
	\section{RF}
	
	\begin{equation}
	\begin{split}
	\text{P}_{RF} &= \int_{A}^{B} \abs{A_{01}}^2 \big(1-\frac{2r^2}{\omega_{0}^2}\big) e^{\frac{-2r^2}{\omega_{0}^2}} 2\pi r dr\\
	&= -\abs{A_{00}}^2 \frac{\pi}{2} \omega_{0}^2 e^{\frac{-2r^2}{\omega_{0}^2}} \bigg(1 + \frac{4 r_4}{\omega_{0}^4} \bigg)  \biggr\rvert_A^B
	\end{split}
	\end{equation}
	
	\begin{equation}
	\text{P}_{in} \\
	= \text{P}_{RF} \biggr\rvert_0^{r_0} \\
	= - \abs{A_{01}}^2  \frac{\pi}{2} \omega_{0}^2 \bigg( e^{\frac{-2r^2}{\omega_{0}^2}} \big(1 + \frac{4 r_4}{\omega_{0}^4} \big) - 1 \bigg)
	\end{equation}
	
	\begin{equation}
	\text{P}_{out} \\
	= \text{P}_{RF} \biggr\rvert_{r_0}^{\infty}\\ 
	= - \abs{A_{01}}^2  \frac{\pi}{2} \omega_{0}^2 e^{\frac{-2r^2}{\omega_{0}^2}} \big(1 + \frac{4 r_4}{\omega_{0}^4} \big) 
	\end{equation}
	
	\begin{equation}
	\text{RF Power Ratio} 
	= \frac{P_{out}}{P_{in}} \\
	= \frac{e^{-2r_0^2/ \omega_{0}^2}} {1 - e^{-2r_0^2/ \omega_{0}^2 }} \approx 2.7844
	\end{equation}
	
	\chapter{Overlap of Gaussian Beams}
	
	Referenced in section
	
	The full Gaussian beam overlap is important in quantitatively defining the amount of power loss obtained when a cavity is mismatched a incoming laser field.
	
	First we define an arbitrary Gaussian beam in cylindrical coordinates:
	
	\begin{equation}
	\begin{split}
	\ket{A(r)} 
	&= \frac{A_0}{q(z)} e^{\frac{-ikr^2}{2q(z)}}\\
	&= \frac{A_0}{q(z)} e^{\frac{-ikr^2(z-iz_0)}{2\abs{q(z)}^2}}
	\end{split}
	\end{equation}
	
	where $A_0$ is a real amplitude, $q(z)= z + i z_0$ is the complex beam parameter, $k$ is the wave number, and $r$ is the radial variable in the transverse direction.

	First we normalize the overlap integral to unity:
	\begin{equation}
	\braket{A(r)|A(r)} 
	=  \frac{\rvert A_0 \rvert^2}{z^2+z_0^2} \int_{0}^{\infty} e^{\frac{-kr^2 z_0}{\abs{q(z)}^2}} 2 \pi r dr = 1
	\end{equation}

	Normalization factor is this:
	\begin{equation}
	A_0 = \sqrt{\frac{k z_0}{\pi}}
	\end{equation}

	For two Gaussian beams with arbitrary q-parameters
	\begin{equation}
	\ket{A_i} = \frac{A_{0,i}}{q_i} e^{ \frac{-ikr^2(z-iz_0)}{2\abs{q_i}^2 }}
	\end{equation}
	where $z_{0,i}$ is the waist size of one particular beam.
	
	The amplitude overlap is:
	\begin{equation}
	\braket{A_1|A_2} = 2 i  \frac{ z_{0,1}z_{0,2}}{q_1 - q_2^*}
	\end{equation}
	
	So the power overlap is:
	\begin{equation}
	\text{Power Overlap} = \vert \braket{A_1|A_2} \vert^2 = 4 \frac{ z_{0,1}z_{0,2}}{\abs{q_1 - q_2^*}^2}
	\end{equation}

	\chapter{Designing Fabry-Perot Cavities as Filters}
	- RT Gouy Phase
	- HOM Coupling
\end{appendices} 

\medskip

\bibliographystyle{unsrt}
\bibliography{Refs}



\end{document}