\documentclass[10pt,a4paper]{book}
\usepackage[utf8]{inputenc}
\usepackage{amsmath}
\usepackage{amsfonts}
\usepackage{amssymb}
\usepackage{graphicx}
\usepackage{xcolor}
\usepackage{listings}
\usepackage{physics}
\usepackage[toc,page]{appendix}
\usepackage{braket}

\title{Adaptive Mode Matching in Advanced LIGO}
\date{Spring 2017}
\author{Thomas Vo}





\begin{document}
		\chapter*{Abstract}
		Going to make LIGO the best possible ever.
	
	\maketitle
		\chapter*{Preface}
		The era of gravitational waves astronomy was ushered in by the LIGO (Laser Interferometer Gravitational-Wave Observatory) collaboration with the detection of a binary black hole collision (Detection paper).  The event that shook the foundation of space-time allowed mankind to view the cosmos in a way that had never been done previously. 
		\addcontentsline{toc}{chapter}{Preface}
	\tableofcontents

\chapter{Introduction}

	\section{Gravitational Waves}\label{gravitational waves}
	In 1915, Albert Einstein published his theory of general relativity.
	
	The seminal equation in this theory is:
	
	\begin{equation} \label{einstein}
	G_{\mu \nu} = 8 \pi T_{\mu \nu}
	\end{equation}
	
	which is a set of 10 coupled second order differential equations that are nonlinear.  In their complete form, equation [1] fully describes the interaction between space-time and mass-energy. To describe the physics in a highly curved space-time, one would have to fully solve the Einstein field equations numerically.  
	
	
	In the weak field approximation, the metric can be described as
	
	\begin{equation} \label{weak}
	g_{\mu \nu}  \approxeq \eta_{\mu \nu} + h_{\mu \nu}
	\end{equation}
	
	where $\eta_{\mu \nu}$ is the metric of flat space time and $|h_{\mu \nu}| \ll 1$ is the perturbation due to a gravitational field.
	
	By plugging in equation \ref{weak} into equation \ref{einstein} and using empty space we obtain the familiar wave equation
	
	\begin{equation} \label{wave}
	\Big(\nabla^2 - \frac{1}{c^2} \pdv[2]{t} \Big) h_{\mu \nu}  = 0
	\end{equation}

	which has a plane-wave solution of the form $h_{\mu \nu} = A_{\mu \nu} e^{ik_{\nu} x^{\nu}}$. 
	
	By using the gauge constraint $h^{\mu \nu}_{,\nu} = 0$, it follows that $A_{\mu \nu} k^{\mu} = 0$ which means that the gravitational wave amplitude is orthogonal to the propagation vector.
	
	By further imposing transverse-traceless gauge, we can constrain the complex amplitude to two orthogonal polarizations which have physical significance:
	
	\begin{equation} \label{gwamp}
	A_{\mu \nu} = 
	\begin{pmatrix}
			0 &    0   &  0      & 0 
		 \\ 0 & A_{xx} &  A_{yx} & 0
		 \\ 0 & A_{xy} & -A_{yy} & 0
		 \\ 0 &    0   &  0      & 0
	\end{pmatrix}
	\end{equation}

	One can ask if a gravitational wave passed by a pair of particles separated by length $L$, what would be the effect on the distance between the two points?  The proper distance is defined as
	
	\begin{equation}\label{propdist}
	\delta l
	= \int{g_{\mu \nu} dx^{\mu} dx^{\nu}} \\
	= \int_{0}^{L}{g_{xx} {d}x}\\
	\approx |g_{xx}(x=0)|^{1/2}\\
	\approx [ 1 + \frac{1}{2} h_{xx}(x=0)] L
	\end{equation} 
	
	which shows us two very important points about the nature of gravitational waves.  Firstly, the effect is very small since the variation in the length is 
	
	Even with the theoretical formulation of gravitational waves resolved, the detection of gravitational waves by ground-based detectors was still a controversial topic among scientists in the field.  This was due to the incredible amount of accuracy needed to actually measure strain. 
	
	Even with some of the most energetic events known to humanity such as the merger of neutron stars and black holes, the amount of strain expected is on the order of 10e-24`
	
	
	
	\section{The LIGO Instrument}
	In the simplest form, the LIGO instrument is an incredibly large Michaelson interferometer.  If we imagine the interferometer as a measure of the differential arm length, it becomes a natural way of detecting gravitational waves.
	
	The LIGO instruments are considered dual-recycled Fabry-Perot interferometers.
	
		\subsection{Simple Michaelson}
		A simple interferometer consists of an input laser, a 50/50 beamsplitter, two mirrors and a photodetector at the output port to read the signal.
		
		In Peter Saulson's book, there is a simple explaination of the light field exiting the anti-symmetric port.
		
		
		As show in section \ref{gravitational waves}, the gravitational wave will modulate the proper length between two free floating points in space. 
	
		\subsection{Fabry-Perot Cavities}\label{FP}
		A Fabry-Perot cavity is an optical system comprised of two mirrors and one laser input. The very simple condition that once the laser has made a round trip around the optics, it is the same shape at the same point.  Conceptually, this may seem simple but in practice, controlling and sensing any optical cavity comes with many challenges.

		Appendix

		References:[Black, Rick's Paper]

		\subsection{Power-Recycled Interferometers}
		Power Recycling
		If the interferometer is operating such that the 4 km arms are exactly different in arms a pi over two times the wavelength, then the intensity of the light at antisymmetric port will be close to null.  This means the power from the arms will reflect back towards the input laser.  Ref[] shows the effect of adding a partially reflecting mirror to increase the optical gain of the Michaelson.
		
		References: [Meers, Kiwamu]
		
		\subsection{Dual-Recycled Interferometers}
		
		
		Figure comparing the three cases of increased sensitivity.
		
		References: [Kiwamu]
		
		\subsection{Noise Budget}
		Ref: Evan Hall, GWINC

		Noise budget:
		- Quantum Noise
		- Seismic
		- Thermal Noise
	
	
	\section{The Effect of a Gravitational Wave on Interferometers}
	
		\subsection{Energy from a Gravitational Wave}
		
		\subsection{Signal to Noise}
		
		\subsection{Range}
	
	\section{Squeezed States of Light}
	The Quantum Noise is a fundamental source that can be improved by modifying the quantum vacuum using correlated photons and injecting the states of light into the antisymmetric port of the interferometer.  Within the LIGO community, this procedure of modifying quantum vacuum is called squeezing. Caves analytically derived the effects of quantum noise on the interferometer as well as the improvement due to i
	
	
	
	References: [Caves, Dwyer, Kwee, Miao]
		
\chapter{The Effects of Mode-Matching}
	Theory section of modematching. Gaussian beams, define relevant quantities, Gouy Phase!
	
	References: [Kognelik and Li]
	
		\subsection{Gaussian Beam Optics}
		When dealing with length sensing degrees of freedoms such as section \ref{FP}, the simple plane wave approximation is sufficient in describing the dynamics, however, when trying to understand the misalignment and mode-mismatch signals, we require Gaussian beam optics and associated higher order modes.
		
		Define a Gaussian beam:
		
		- Consider the famous Maxwell's equations in vacuum:
		
		\begin{equation}
		\label{18.1:1}
		\begin{aligned}
		 \nabla \times \vec{E} &=-\frac{\partial \mathbf{B}} {\partial t},&
		\\\nabla \cdot \vec{B} &=0&,
		\\\nabla \times \mathbf{B} &= \mu\ \mathbf{J} + \frac{1}{c^2} \frac{\partial \mathbf{E}} {\partial t}&
		\\
		\nabla \cdot \mathbf{E} &= \frac{\rho}{\epsilon}&
		\end{aligned}
		\end{equation}
		
		- Concentrating on the electric field in vacuum, we arrive at the Helmholtz Equation
		
		\begin{equation}
		(\nabla^2 + k^2 ) E(x,y,z) = 0
		\end{equation}
		
		If we apply the paraxial approximations,
		
		\begin{subequations}
		\begin{equation}\label{paraxiala}
		\bigg| { \partial^2 u \over \partial z^2 } \bigg|  \ll  \bigg| { k {\partial u \over \partial z} } \bigg|
		\end{equation}
		\begin{equation}\label{paraxialb}
		\bigg| { \partial^2 u \over \partial z^2 } \bigg|  \ll  \bigg| { k {\partial^2 u \over \partial x^2} } \bigg|
		\end{equation}
		\begin{equation}\label{paraxialc}
		\bigg| { \partial^2 u \over \partial z^2 } \bigg|  \ll  \bigg| { k {\partial^2 u \over \partial y^2} } \bigg|
		\end{equation}
		\end{subequations}
		
		
		- Now get the Paraxial Helmholtz Equation, with the transverse Laplacian operator:
		
		\begin{equation}
		U(r) = A(r) e^{-ikz)}
		\end{equation}
		
		\begin{equation}
		\nabla_T^2 A(r) - i2k {\partial A(r) \over \partial z} = 0
		\end{equation}
		
		where $\nabla_T^2 = {\partial^2  \over \partial x^2} + {\partial^2  \over \partial y^2} $ is the transverse Laplacian 
		
		- Solution to the Paraxial Helmholtz equation:
		
		\begin{equation} \label{complexenvelope}
		A(r) = \frac{A_0}{q(z)} e^{\frac{-ikr^2}{2q(z)}} , \quad q(z)=z+iz_0
		\end{equation}
		
		where $z_0$ is the Rayleigh range and is directly proportional to the square of the waist size.  It is useful to rewrite $q(z)$ as
		
		\begin{equation}
		\frac{1}{q(z)} = \frac{1}{R(z)} - i \frac{\lambda}{\pi W^2(z)}
		\end{equation} 
		
		- Complex amplitude * phase
		
		\begin{equation}
		U(r) = A_0 \frac{W_0}{W(z)} e^{-\frac{r^2}{W^2(z)}} e^{-ikz - ik \frac{r^2}{2R(z)} + i \phi(z)}
		\end{equation}
		
		- Waist size and Beam Size
		\begin{subequations}
		\begin{equation}
		W(z) = W_0 \sqrt{1 + \bigg( \frac{z}{z_0} \bigg)^2}
		\end{equation}
		\begin{equation}
		R(z) = z \bigg[ 1 + \bigg( \frac{z}{z_0} \bigg)^2 \bigg]
		\end{equation}
		
		\end{subequations}



		- Radius of Curvature
		
		- Gouy Phase + graph
		
		- Intensity
		
		- HOM equation for LG and HG
		
		\subsection{Coupling Power to Higher Order Modes}
		
		In general, there are a number of ways that a TEM00 mode can couple into higher order modes.  
		
		This can be broken up into two categories: mode misalignment and mode mismatch.  Everyone says it differently.
		
		Summarize the coupling [Anderson]
		
		Summarize the formalism with ket notation as a sum of the eigenmodes of a cavity.
		References:[Anderson, Guido]
		
		
			\subsubsection{Misalignment}
			Mode Misalignment: tilted axis, displaced axis
			References: [Daniel and Nergis, own calculations]
				
			\subsubsection{Mode Mismatch}
			References: [Own Calculations]
				
		
		\subsection{Wavefront Sensing}
		- General Formulation
			- Modulation and Demodulation
			
		- 
		
		
		

		\subsection{Effects of Mode-matching on Squeezing}
		Still not clear to me.
		In the most simple sense, a loss is akin to coupling quantum vacuum into a squeezed state.
		References: [Miao, Sheon?]
	
	
%%%%%%%%%%%%%%%%%%%%%%	
\chapter{Modeling Mode-Matching}	
	\section{How it works}
	Summary of how Finesse works (input output matrix), how it handles HOMs

	\section{Finesse Simulations}
		\subsection{ALIGO Design with FC and Squeezer}
		\subsection{Looking at just Modal Change}
		\subsection{QM Limited Sensitivity}
			
	\section{Results}
		* Signal recycling cavity mismatches

		* Mismatches before the OMC
			
		* Mismatch contour graph: Comparing all of ALIGO cavities
			
		* Optical Spring pops up at 7.4 Hz in the Signal-to-Darm TF, re-run with varying SRM Trans which should.

%%%%%%%%%%%%%%%%%%%%%%
\chapter{Mode Matching Cavities at Syracuse}
In conjunction with Sandoval et al, the adaptive modematching table top experiment was able to show the feasibility of a fully dynamical system.
	\section{Adaptive Mode Matching}
	Real time digital system and model.
	
	Signal chain.
	
	\section{Actuators}
		\subsection{Thermal Lenses}
		Fabian's work and UFL paper.
		\subsection{Translation Stages}
		
	\section{Sensors}
		\subsection{Bullseye Photodiodes}
		As stated in section [], we saw that the error signal for a mode-mismatched cavity has cylindrical symmetry due to the beating between the LG-01 and the LG-00 mode.  This means that the quadrant photodiodes would not have a way to detect the modal content of a cavity.  One way to solve this is to introduce a type of detector that can sense the power outside versus the power inside (ugh re write this).
		
		In [Rana's IO final design document] 
		
		
		Why do we need bullseye photodiodes, meantion the geometry of the error signal.
		
		Derivations in the appendix.
		
		Picture of BPD
		
		Pitch and Yaw sensing matrix
		
		Explain why we need $\omega_{0} = \sqrt{2} r_0$
		
		The ratio of the out over in will give:
		
		\begin{equation}
		\text{Power Ratio} = \frac{\text{P}_2 + \text{P}_3 + \text{P}_4}{\text{P}_1}  \\
		= \frac{e^{-2r_0^2/ \omega_{0}^2}} {1 - e^{-2r_0^2/ \omega_{0}^2 }} \approx 0.582
		\end{equation}
		
		\subsection{Mode Converters}
		
		
		

%%%%%%%%%%%%%%%%%%%%%%
\chapter{Mode Matching Cavities at LIGO Hanford}

	\section{Active Wavefront Control System}
	
	\section{SRC}
	The importance of mode-matching actually goes beyond reducing the amount of losses in coupled cavities.  It also is important for cavity stability.  If we look at the g-factor of a cavity, it is required through ABCD transformations that the values lay between 0 and 1.  For the signal recycling cavity, if the round trip gouy phase is off by a few millimeters, the stability of the cavity can be compromised. 
	
	
	\section{Beam Jitter}
	
	Current measurements of mode-matching.

%%%%%%%%%%%%%%%%%%%%%%

\chapter{Solutions for Detector Upgrades}

* SR3 Heater

* SRM Heater

* Bullseye photodetectors

* Operation: range (in terms of watts and % percentage mismatch)

* Translation stages

* Mechanical description (Solidworks designs)

* Constraints (range, vacuum, alignment, integration)

* Electronics 

* Software


	\listoffigures
	\listoftables

\begin{appendices}
	
	\chapter{Resonator Formulas}
	
	
	Ray Trace: Round trip phase, cavity stability
	
	TF: analytical derivation
	
	Important quantities: Finesse, cavity pole, free spectral range
	
	Effect of higher order modes into the cavity, mode scanning. All comes from round trip Gouy phase.
	
	
	\chapter{Bullseye Photodiode Characterization}
	
	\section{DC}
	\begin{equation}
	\begin{split}
	\text{Power} &= \int_{A}^{B} \abs{A_{00}}^2e^{\frac{-2r^2}{\omega_{0}^2}} 2\pi r dr\\
			&= -\abs{A_{00}}^2 \frac{\pi \omega_{0}^2}{2} e^{\frac{-2r^2}{\omega_{0}^2}} \biggr\rvert_A^B
	\end{split}
	\end{equation}
	
	\begin{equation}
	\text{P}_{in} = \text{Power} \biggr\rvert_0^{r_0} = \abs{A_{00}}^2 \frac{\pi \omega_{0}^2}{2} [1 - e^{\frac{-2r_0^2}{\omega_{0}^2}}]
	\end{equation}
	
	\begin{equation}
	\text{P}_{out} = \text{Power} \biggr\rvert_{r_0}^{\infty} = \abs{A_{00}}^2 \frac{\pi \omega_{0}^2}{2} [e^{\frac{-2r_0^2}{\omega_{0}^2}}]
	\end{equation}
	
	\begin{equation}
	\text{P}_{total} =  \text{P}_{in} + \text{P}_{out}
	\end{equation}
	
	\begin{equation}
	\omega = \sqrt{\frac{\text{P}_{total}}{\abs{A_{00}}^2 \pi / 2}}
	\end{equation}
	
	\begin{equation}
	\text{DC Power Ratio} 
	= \frac{P_{out}}{P_{in}} \\
	= \frac{e^{-2r_0^2/ \omega_{0}^2}} {1 - e^{-2r_0^2/ \omega_{0}^2 }} \approx 0.582
	\end{equation}
	
	\section{RF}
	
	\begin{equation}
	\begin{split}
	\text{P}_{RF} &= \int_{A}^{B} \abs{A_{01}}^2 \big(1-\frac{2r^2}{\omega_{0}^2}\big) e^{\frac{-2r^2}{\omega_{0}^2}} 2\pi r dr\\
	&= -\abs{A_{00}}^2 \frac{\pi}{2} \omega_{0}^2 e^{\frac{-2r^2}{\omega_{0}^2}} \bigg(1 + \frac{4 r_4}{\omega_{0}^4} \bigg)  \biggr\rvert_A^B
	\end{split}
	\end{equation}
	
	\begin{equation}
	\text{P}_{in} \\
	= \text{P}_{RF} \biggr\rvert_0^{r_0} \\
	= - \abs{A_{01}}^2  \frac{\pi}{2} \omega_{0}^2 \bigg( e^{\frac{-2r^2}{\omega_{0}^2}} \big(1 + \frac{4 r_4}{\omega_{0}^4} \big) - 1 \bigg)
	\end{equation}
	
	\begin{equation}
	\text{P}_{out} \\
	= \text{P}_{RF} \biggr\rvert_{r_0}^{\infty}\\ 
	= - \abs{A_{01}}^2  \frac{\pi}{2} \omega_{0}^2 e^{\frac{-2r^2}{\omega_{0}^2}} \big(1 + \frac{4 r_4}{\omega_{0}^4} \big) 
	\end{equation}
	
	\begin{equation}
	\text{RF Power Ratio} 
	= \frac{P_{out}}{P_{in}} \\
	= \frac{e^{-2r_0^2/ \omega_{0}^2}} {1 - e^{-2r_0^2/ \omega_{0}^2 }} \approx 2.7844
	\end{equation}
	
	\chapter{Overlap of Gaussian Beams}
	
	Referenced in section
	
	The full Gaussian beam overlap is important in quantitatively defining the amount of power loss obtained when a cavity is mismatched a incoming laser field.
	
	First we define an arbitrary Gaussian beam in cylindrical coordinates:
	
	\begin{equation}
	\begin{split}
	\ket{A(r)} 
	&= \frac{A_0}{q(z)} e^{\frac{-ikr^2}{2q(z)}}\\
	&= \frac{A_0}{q(z)} e^{\frac{-ikr^2(z-iz_0)}{2\abs{q(z)}^2}}
	\end{split}
	\end{equation}
	
	where $A_0$ is a real amplitude, $q(z)= z + i z_0$ is the complex beam parameter, $k$ is the wave number, and $r$ is the radial variable in the transverse direction.

	First we normalize the overlap integral to unity:
	\begin{equation}
	\braket{A(r)|A(r)} 
	=  \frac{\rvert A_0 \rvert^2}{z^2+z_0^2} \int_{0}^{\infty} e^{\frac{-kr^2 z_0}{\abs{q(z)}^2}} 2 \pi r dr = 1
	\end{equation}

	Normalization factor is this:
	\begin{equation}
	A_0 = \sqrt{\frac{k z_0}{\pi}}
	\end{equation}

	For two Gaussian beams with arbitrary q-parameters
	\begin{equation}
	\ket{A_i} = \frac{A_{0,i}}{q_i} e^{ \frac{-ikr^2(z-iz_0)}{2\abs{q_i}^2 }}
	\end{equation}
	where $z_{0,i}$ is the waist size of one particular beam.
	
	The power overlap is:
	\begin{equation}
	\text{Overlap} = \braket{A_1|A_2} = 4 \frac{ \sqrt{z_{0,1}z_{0,2}}}{\abs{q_1 - q_2^*}^2}
	\end{equation}

\end{appendices} 


\end{document}