\documentclass[oneside]{book}
\usepackage[margin=1.0in]{geometry}
\usepackage{setspace}
\usepackage[utf8]{inputenc}
\usepackage{amsmath}
\usepackage{amsfonts}
\usepackage{amssymb}
\usepackage{graphicx}
\usepackage{xcolor}
\usepackage{listings}
\usepackage{physics}
\usepackage[toc,page]{appendix}
\usepackage{braket}
\usepackage[pagebackref]{hyperref}   


\title{Adaptive Mode Matching in Advanced LIGO}
\date{Spring 2017}
\author{Thomas Vo}


\begin{document}
	
\doublespacing
		\chapter*{Abstract}
		Going to make LIGO the best possible ever.
	
	\maketitle
		\chapter*{Preface}
		The era of gravitational waves astronomy was ushered in by the LIGO (Laser Interferometer Gravitational-Wave Observatory) collaboration with the detection of a binary black hole collision (Detection paper).  The event that shook the foundation of space-time allowed mankind to view the cosmos in a way that had never been done previously. 
		\addcontentsline{toc}{chapter}{Preface}
	\tableofcontents

\chapter{Introduction}

	\section{Gravitational Waves}\label{gravitational waves}
	In 1915, Albert Einstein published his theory of general relativity.
	
	The seminal equation in this theory is:
	
	\begin{equation} \label{einstein}
	G_{\mu \nu} = 8 \pi T_{\mu \nu}
	\end{equation}
	
	which is a set of 10 coupled second order differential equations that are nonlinear.  In their complete form, equation [1] fully describes the interaction between space-time and mass-energy. To describe the physics in a highly curved space-time, one would have to fully solve the Einstein field equations numerically.  
	
	
	In the weak field approximation, the metric can be described as
	
	\begin{equation} \label{weak}
	g_{\mu \nu}  \approxeq \eta_{\mu \nu} + h_{\mu \nu}
	\end{equation}
	
	where $\eta_{\mu \nu}$ is the metric of flat space time and $|h_{\mu \nu}| \ll 1$ is the perturbation due to a gravitational field.
	
	By plugging in equation \ref{weak} into equation \ref{einstein} and using empty space we obtain the familiar wave equation
	
	\begin{equation} \label{wave}
	\Big(\nabla^2 - \frac{1}{c^2} \pdv[2]{t} \Big) h_{\mu \nu}  = 0
	\end{equation}

	which has a plane-wave solution of the form $h_{\mu \nu} = A_{\mu \nu} e^{ik_{\nu} x^{\nu}}$. 
	
	By using the gauge constraint $h^{\mu \nu}_{,\nu} = 0$, it follows that $A_{\mu \nu} k^{\mu} = 0$ which means that the gravitational wave amplitude is orthogonal to the propagation vector.
	
	By further imposing transverse-traceless gauge, we can constrain the complex amplitude to two orthogonal polarizations which have physical significance:
	
	\begin{equation} \label{gwamp}
	A_{\mu \nu} = 
	\begin{pmatrix}
			0 &    0   &  0      & 0 
		 \\ 0 & A_{xx} &  A_{yx} & 0
		 \\ 0 & A_{xy} & -A_{yy} & 0
		 \\ 0 &    0   &  0      & 0
	\end{pmatrix}
	\end{equation}

	Add Figure of GWs

	It is natural to attempt to understand the physical interpretation of Eq.\ref{gwamp} as an affect on the position of a free floating particle. Consider the four-velocity, $U^{\alpha}$, in the transverse traceless gauge where the coordinate itself is attached the particles and incorporates any small wiggles that would shake the coordinates.  Of course, any free particles will follow the geodesic equation
	
	\begin{equation}\label{geodesic}
	\nabla U^{\alpha} = \frac{\text{d}}{\text{d} \tau} U^{\alpha} + \Gamma^{\alpha}_{\mu \nu} U^{\mu} U^{\nu} = 0
	\end{equation}	
	where $\Gamma^{\alpha}_{\mu \nu} = \frac{1}{2} g^{\gamma \alpha}(g_{\gamma \mu, \nu} + g_{\gamma \nu,\mu} - g_{\mu \nu, \gamma} )$ are the famous Christoffel symbols.
	By evaluating the first term of the acceleration in Eq.\ref{geodesic},
	\begin{equation}\label{accel}
	\bigg(\frac{\text{d}U^{\alpha}}{\text{d}\tau}\bigg)_0 = -\Gamma^{\alpha}_{00} 
	\\ = \frac{1}{2} \eta_{\mu \nu} (h_{\beta 0, 0} + h_{0 \beta, 0} + h_{0 0, \beta} )
	\end{equation}
	However, comparing Eq.\ref{gwamp} and Eq.\ref{accel}, it is clear that if the particle is initially at rest, then a moment later it is still at rest! The term "at rest" is actually used liberally here since the coordinate system varies along with the gravitational wave. 
	
	Alternatively, one can ask if a gravitational wave passed by a pair of particles separated by length $L$, what would be the effect on the distance between two points?  The proper distance is defined as

	\begin{equation}\label{propdist}
	\delta l
	= \int{g_{\mu \nu} dx^{\mu} dx^{\nu}} \\
	= \int_{0}^{L}{g_{xx} {d}x}\\
	\approx |g_{xx}(x=0)|^{1/2}\\
	\approx [ 1 + \frac{1}{2} h_{xx}(x=0)] L
	\end{equation} 
	
	which shows us two very important points about the nature of gravitational waves.  Firstly, the effect is very small since the length variation is, by definition, a small perturbation in flat space-time.  Secondly, the effect is proportional to the initial separation between the particles. This means a detector which is large will have a better chance to measure these small effects, a point that drove the design of the Laser Interferometer Gravitational-Wave Interferometer (LIGO).
	
	\subsection{Measuring Gravitational Waves with Light}
	Even with the theoretical formulation of gravitational waves resolved by the 1970s, detection of GWs by ground-based detectors was still a controversial topic among scientists in the field.  This was due to the incredible amount of accuracy needed to actually measure strain. 
	
	\subsection{Energy from a Gravitational Wave}
	
	\subsection{Signal to Noise}
	
	\subsection{Range}
	\subsection{Sources of Gravitational Waves}
	\subsubsection{Compact Binary Inspirals}
	\subsubsection{Continuous Waves}
	\subsubsection{Bursts}
	\subsubsection{Stochastic}
	Even with some of the most energetic events known to humanity such as the merger of neutron stars and black holes, the amount of strain expected is on the order of 10e-24.
	
	\cite{Saulson}
	
	\subsection{Measuring Gravitational Waves with Light}
	Even with the theoretical formulation of gravitational waves resolved by the 1970s, detection of GWs by ground-based detectors was still a controversial topic among scientists in the field.  This was due to the incredible amount of accuracy needed to actually measure strain. 
	
	\section{The LIGO Instrument}
	In the simplest form, the LIGO instrument is an incredibly large Michaelson interferometer.  If we imagine the output as a measure of the differential arm length, it becomes a natural way of detecting gravitational waves.
	
	The LIGO instruments are considered dual-recycled Fabry-Perot interferometers.
	
		\subsection{Simple Michaelson}
		A simple interferometer consists of an input laser, a 50/50 beamsplitter, two mirrors and a photodetector at the output port to read the signal.
		
		In Peter Saulson's book, there is a simple explaination of the light field exiting the anti-symmetric port.
		
		
		As show in section \ref{gravitational waves}, the gravitational wave will modulate the proper length between two free floating points in space. 
	
		\subsection{Fabry-Perot Cavities}\label{FP}
		A Fabry-Perot cavity is an optical system comprised of two mirrors and one laser input. The very simple condition that once the laser has made a round trip around the optics, it is the same shape at the same point.  Conceptually, this may seem simple but in practice, controlling and sensing any optical cavity comes with many challenges.

		Appendix

		References:[Black, Rick's Paper]

		\subsection{Power-Recycled Interferometers}
		Power Recycling
		If the interferometer is operating such that the 4 km arms are exactly different in arms a pi over two times the wavelength, then the intensity of the light at antisymmetric port will be close to null.  This means the power from the arms will reflect back towards the input laser.  Ref[] shows the effect of adding a partially reflecting mirror to increase the optical gain of the Michaelson.
		
		References: [Meers, Kiwamu]
		
		\subsection{Dual-Recycled Interferometers}
		
		
		Figure comparing the three cases of increased sensitivity.
		
		References: [Kiwamu]
		
		\subsection{Fundamental Noise Sources}
		Ref: Evan Hall, GWINC

		Noise budget:

		\subsubsection{Quantum Noise}
		\subsubsection{Seismic Noise}
		\subsubsection{Thermal Noise}
		\subsubsection{Newtonian Noise}

	
	
	\section{Squeezed States of Light}
	The Quantum Noise is a fundamental source that can be improved by modifying the quantum vacuum using correlated photons and injecting the states of light into the antisymmetric port of the interferometer.  Within the LIGO community, this procedure of modifying quantum vacuum is called squeezing. Caves analytically derived the effects of quantum noise on the interferometer as well as the improvement due to i
	
	References: [Caves, Dwyer, Kwee, Miao]
		
\chapter{The Fundamentals of Mode-Matching}
	Theory section of modematching. Gaussian beams, define relevant quantities, Gouy Phase!
	
	In order to properly explain the effects of electromagnetic waves in the LIGO interferometers, it is useful to review the 
	
	References: [Kognelik and Li]
	
		\section{Gaussian Beam Optics}
		When dealing with length sensing degrees of freedoms such as section \ref{FP}, the simple plane wave approximation is sufficient in describing the dynamics, however, when trying to understand the misalignment and mode-mismatch signals, it is necessary to incorporate Gaussian beam optics and associated their higher order modes.

		
		Consider the famous Maxwell's equations in vacuum:
		
		\begin{equation}
		\label{18.1:1}
		\begin{aligned}
		 \nabla \times \mathbf{E} &=-\frac{\partial \mathbf{B}} {\partial t},&
		\\\nabla \cdot \mathbf{B} &=0&,
		\\\nabla \times \mathbf{B} &= \mu\ \mathbf{J} + \frac{1}{c^2} \frac{\partial \mathbf{E}} {\partial t}&
		\\
		\nabla \cdot \mathbf{E} &= \frac{\rho}{\epsilon}&
		\end{aligned}
		\end{equation}
		
		
		Concentrating on the electric field in vacuum, we arrive at the Helmholtz Equation
		
		\begin{equation}\label{Helmholtz}
		(\nabla^2 + k^2 ) \mathbf{U}(\mathbf{r},t) = 0
		\end{equation}
	
		
		where $k=\frac{2\pi\nu}{c}$ is the wave number and $\mathbf{U}(\mathbf{r},t)$ is the complex amplitude which can describe either the electric or magnetic fields.  
		
		It is possible to express the solution to Eq. \ref{Helmholtz} as
		a plane wave with a modulated complex envelope
		\begin{equation}
		\mathbf{U}(\mathbf{r}) = \mathbf{A}(\mathbf{r}) e^{-ikz)}
		\end{equation}
		
		By imposing the constraints which force the envelope to vary slowly with respect to the z-axis within the distance of one wavelength $\lambda = 2\pi/k$,

		\begin{subequations}
		\begin{equation}\label{paraxiala}
		\bigg| { \partial^2 \mathbf{A} \over \partial z^2 } \bigg|  \ll  \bigg| { k {\partial \mathbf{A} \over \partial z} } \bigg|
		\end{equation}
		\begin{equation}\label{paraxialb}
		\bigg| { \partial^2 \mathbf{A} \over \partial z^2 } \bigg|  \ll  \bigg| { k {\partial^2 \mathbf{A} \over \partial x^2} } \bigg|
		\end{equation}
		\begin{equation}\label{paraxialc}
		\bigg| { \partial^2 \mathbf{A} \over \partial z^2 } \bigg|  \ll  \bigg| { k {\partial^2 \mathbf{A} \over \partial y^2} } \bigg|
		\end{equation}
		\end{subequations}
		
		the partial differential equation which arises is called the Paraxial Helmholtz Equation:
		
		\begin{equation}\label{paraHelmholtz}
		\nabla_T^2 A(r) - i2k {\partial A(r) \over \partial z} = 0
		\end{equation}
		
		where $\nabla_T^2 = {\partial^2  \over \partial x^2} + {\partial^2  \over \partial y^2} $ is the transverse Laplacian.  A simple solution for Eq.\ref{paraHelmholtz} is the complex paraboloidal wave
		
		\begin{equation} \label{complexenvelope}
		A(\mathbf{r}) = \frac{A_0}{q(z)} e^{\frac{-ikr^2}{2q(z)}} , \quad q(z)=z+iz_0
		\end{equation}
		
		where $z_0$ is the Rayleigh range and is directly proportional to the square of the waist size.  In order to separate the amplitude and phase portions of the wave, it is useful to rewrite $q(z)$ as
		
		\begin{equation}\label{invq}
		\frac{1}{q(z)} = \frac{1}{R(z)} - i \frac{\lambda}{\pi W^2(z)}
		\end{equation} 
		
		Plugging Eq.\ref{invq} into \ref{complexenvelope} leads directly to the complex amplitude for a Gaussian Beam
		
		\begin{equation}
		U(r) = A_0 \frac{W_0}{W(z)} e^{-\frac{r^2}{W^2(z)}} e^{-ikz - ik \frac{r^2}{2R(z)} + i \phi(z)}
		\end{equation}
		
		where
		\begin{subequations}
		\begin{equation}
		W(z) = W_0 \sqrt{1 + \bigg( \frac{z}{z_0} \bigg)^2}
		\end{equation}
		\begin{equation}
		R(z) = z \bigg[ 1 + \bigg( \frac{z}{z_0} \bigg)^2 \bigg]
		\end{equation}
		\begin{equation}
		\phi(z)= \text{tan}^{-1}\bigg(\frac{z}{z_0}\bigg)
		\end{equation}
		\begin{equation}
		W_0 = \sqrt{\frac{\lambda z_0}{\pi}}
		\end{equation}
		\end{subequations}

		
		- Gouy Phase + graph
		
		- Intensity and Power
		
		\subsubsection{Hermite-Gauss Modes}
		The fundamental Gaussian beam is not the only solution which can be used to solve Eq.\ref{paraHelmholtz}.  In fact, there exists a complete set of solutions that can solve the paraxial Helmholtz Equation, which are referred to as the higher order modes (HOMs)
		\begin{equation}\label{HG}
		\begin{aligned}
		&U_{mn}(x,y,z) = A_{mn}\bigg[ \frac{W_0}{W(z)} \bigg] \mathbb{G}_m\Bigg( \frac{\sqrt{2}x}{W(z)}  \Bigg) \mathbb{G}_n\Bigg( \frac{\sqrt{2}y}{W(z)} \Bigg)\\
		&\times \text{exp} \bigg\{ -ikz - \frac{ik(x^2+y^2)}{2R(z)} + i(m+n+1)\phi(z) \bigg\}
		\end{aligned}
		\end{equation}
		
		where,
		\begin{equation}
		\mathbb{G}_(u) = \mathbb{H}(u) \, \text{exp}(-u^2/2)
		\end{equation}
		
		and $ \mathbb{H}(u)$ are the well known Hermite polynomials.  It is important to mention that the Gouy phase of the complex amplitude is different than the fundamental Gaussian beam by a factor of $(m + n + 1)$. Also, the intensity distribution of these higher order modes are much different. Both of these facts will become extremely important in the following wavefront sensing discussion.
		
		\subsubsection{Laguerre Modes}
		Another complete set of alternative solutions to Eq.\ref{paraHelmholtz} exists which are called the Laguerre-Gauss modes
		
		\begin{equation}\label{LG}
		\begin{aligned}
		&V_{\mu\nu}(\rho,\theta,z) = A_{\mu\nu}\bigg[ \frac{W_0}{W(z)} \bigg] \mathbb{L}^{\mu}_{\nu} \Bigg( \frac{\sqrt{2}x}{W(z)}  \Bigg) \\
		&\times \text{exp} \bigg\{-ikz-\frac{ik\rho^2}{2R(z)} + i(\mu+2\nu+1)\phi(z) \bigg\}
		\end{aligned}
		\end{equation}
		
		where $\mathbb{L}^{\mu}_{\nu} \Bigg( \frac{\sqrt{2}x}{W(z)}  \Bigg)$ is the Laguerre polynomial function. Both equations \ref{HG} and \ref{LG} are able to fully describe any complex electromagnetic amplitude; and because they both form complete sets, there is a rotation which can map from one basis to the other [ref Bond and Biejergasern]
		
		\begin{equation}
		U^{LG}_{\mu \nu} (x,y,z) = \sum\limits_{k}^{N} i^k b(n,m,k) U^{HG}_{N-k,k} (x,y,z)
		\end{equation}
		where
		\begin{equation}
		b(n,m,k) = \sqrt{\bigg( \frac{(N-k)!k!}{2^N n!m!} \bigg)} \frac{1}{k!} \frac{\text{d}^k}{\text{d}t^k}[(1-t)^m (1+t)^m]\vert_{t=0}
		\end{equation}

		\subsection{Misalignment and Mode Mismatch}\label{Misalignment}
		Morrison and Anderson derived a simplistic way of how small misalignments and mismodematched cavities can couple the fundamental Gaussian beam mode into various higher order modes.
		
		Consider Eq.\ref{HG} in a more simplified form:
		
		
		
		
		Mode Misalignment: tilted axis, displaced axis
		References: [Daniel and Nergis, own calculations]
			
		
		
		Figure: Modematching
			References: [Own Calculations]
				
		
		\section{Wavefront Sensing}
		Heterodyne detection using a modal decomposition of the full electric field allows the use of wavefront sensors to extract an error signal from the optical system.  Hefetz et.al Ref[Sigg and Nergis] created a formalism to describe the use of wavefront sensors by creating frequency sidebands which accumulate a different Gouy phase than the electric field at the carrier frequency when passed through the optical system.  
		
		Consider a general equation for an electric field which is a linear combination of all higher order modes of the complex amplitude
		\begin{equation}
		E(x,y,z) = \sum\limits_{m,n}^{\infty} a_{mn} U_{mn}(x,y,z)
		\end{equation}
		
		where $ U_{mn}(x,y,z)$ are the eigenmodes of the described in Eq[] and $a_{mn}$ is the complex amplitude.  It is also convenient in the following analysis to use vectors when describing the composition of the electric fields.
		
		\begin{equation}
		\ket{E(x,y,z)} = \begin{pmatrix} E_{00} \\ E_{01} \\ E_{10} \\ \vdots \end{pmatrix}
		\end{equation}

		\begin{equation} \label{misalign_matrix}
		\hat{\Theta}_{\mu \nu} = 
		\begin{pmatrix}
		   1			&2i\theta_x		&2i\theta_y		& 0 & 0
		\\ 2i\theta_x	&1				&0				& 0	& 0
		\\ 2i\theta_y	&0				&1				& 0	& 0
		\\ 0			&0				&0				& 1	& 0
		\\ 0			&0				&0				& 0	& 1
		\end{pmatrix}
		\end{equation}

		\begin{equation} \label{mistrans_matrix}
		\hat{\mathbb{D}}_{\mu \nu} = 
		\begin{pmatrix}
			1					&\alpha_x/\omega_{0}	&\alpha_y/\omega_{0}	& 0 & 0
		\\ \alpha_x/\omega_{0}	&1						&0						& 0 & 0
		\\ \alpha_y/\omega_{0}	&0						&1						& 0	& 0
		\\ 0					&0						&0						& 1	& 0
		\\ 0					&0						&0						& 0 & 1 
		\end{pmatrix}
		\end{equation}
		
		
		\begin{equation} \label{waistloc_matrix}
		\hat{\mathbb{Z}}_{\mu \nu} = 
		\begin{pmatrix}
		1				&0		&0		&\Delta z_x 	&\Delta z_y  
		\\ 0			&1		&0		&0 				&0
		\\ 0			&0		&1		&0 				&0
		\\ \Delta z_x	&0		&0		&1 				&0
		\\ \Delta z_y	&0		&0		&0				&1 
		\end{pmatrix}
		\end{equation}
		
		where $\Delta z_{(x,y)} =  \frac{i}{\sqrt{2}} \frac{\lambda b}{2\pi\omega_{0}} $
		
		
		\begin{equation} \label{waistsize_matrix}
		\hat{\mathbb{Z}}_{0, \mu \nu} = 
		\begin{pmatrix}
		1					&0		&0		&\Delta z_{0,x} 	&\Delta z_{0,y} 
		\\ 0				&1		&0		&0 					&0
		\\ 0				&0		&1		&0 					&0
		\\ \Delta z_{0,x} 	&0		&0		&1 					&0
		\\ \Delta z_{0,y} 	&0		&0		&0					&1 
		\end{pmatrix}
		\end{equation}

		where $\Delta z_{0,(x,y)} =   \frac{1}{\sqrt{2}} \frac{\omega'-\omega_{0}}{\omega_{0}} $

		Figure: Beat frequency between two modes.

		\section{Effects of Mode-matching on Squeezing}
		Still not clear to me.
		In the most simple sense, a loss is akin to coupling quantum vacuum into a squeezed state.
		References: [Miao, Sheon, Kimble, Dwyer]
	
	
%%%%%%%%%%%%%%%%%%%%%%	
e\chapter{Simulating Mode-Matching with Finesse}	
	\section{How it works}
	Summary of how Finesse works (input output matrix), how it handles HOMs
	\section{Finesse Simulations}
		\subsection{ALIGO Design with FC and Squeezer}
		\subsection{Looking at just Modal Change}
		\subsection{QM Limited Sensitivity}
			
	\section{Results}
		* Signal recycling cavity mismatches

		* Mismatches before the OMC
			
		* Mismatch contour graph: Comparing all of ALIGO cavities
		
		* Optical Spring pops up at 7.4 Hz in the Signal-to-Darm TF, re-run with varying SRM Trans which should.

%%%%%%%%%%%%%%%%%%%%%%
\chapter{Experimental Mode Matching Cavities at Syracuse}
In conjunction with Sandoval et al, the adaptive modematching table top experiment was able to show the feasibility of a fully dynamical system.
	\section{Adaptive Mode Matching}
	Real time digital system and model.
	
	Signal chain.
	
	\section{Actuators}
		\subsection{Thermal Lenses}
		Fabian's work and UFL paper.
		\subsection{Translation Stages}
		
	\section{Sensors}
		\subsection{Bullseye Photodiodes}
		As stated in section [], we saw that the error signal for a mode-mismatched cavity has cylindrical symmetry due to the beating between the LG-01 and the LG-00 mode.  This means that the quadrant photodiodes would not have a way to detect the modal content of a cavity.  One way to solve this is to introduce a type of detector that can sense the power outside versus the power inside (ugh re write this).
		
		In [Rana's IO final design document] 
		
		
		Why do we need bullseye photodiodes, meantion the geometry of the error signal.
		
		Derivations in the appendix.
		
		Picture of BPD
		
		Pitch and Yaw sensing matrix
		
		Explain why we need $\omega_{0} = \sqrt{2} r_0$
		
		The ratio of the out over in will give:
		
		\begin{equation}
		\text{Power Ratio} = \frac{\text{P}_2 + \text{P}_3 + \text{P}_4}{\text{P}_1}  \\
		= \frac{e^{-2r_0^2/ \omega_{0}^2}} {1 - e^{-2r_0^2/ \omega_{0}^2 }} \approx 0.582
		\end{equation}
		
		\subsection{Mode Converters}
		
		
		

%%%%%%%%%%%%%%%%%%%%%%
\chapter{Mode Matching Cavities at LIGO Hanford}

	\section{Active Wavefront Control System}
	
	\section{SRC}
	The importance of mode-matching actually goes beyond reducing the amount of losses in coupled cavities.  It also is important for cavity stability.  If we look at the g-factor of a cavity, it is required through ABCD transformations that the values lay between 0 and 1.  For the signal recycling cavity, if the round trip gouy phase is off by a few millimeters, the stability of the cavity can be compromised. 
	
	
	\section{Beam Jitter}
	
	Current measurements of mode-matching.

%%%%%%%%%%%%%%%%%%%%%%

\chapter{Solutions for Detector Upgrades}

* SR3 Heater

* SRM Heater

* Bullseye photodetectors

* Operation: range (in terms of watts and % percentage mismatch)

* Translation stages

* Mechanical description (Solidworks designs)

* Constraints (range, vacuum, alignment, integration)

* Electronics 

* Software


	\listoffigures
	\listoftables

\begin{appendices}

	\chapter{Resonator Formulas}
	
	
	Ray Trace: Round trip phase, cavity stability
	
	TF: analytical derivation
	
	Important quantities: Finesse, cavity pole, free spectral range
	
	Effect of higher order modes into the cavity, mode scanning. All comes from round trip Gouy phase.
	
	
	\chapter{Higher Order Mode Coupling}
	In general, there are a number of ways that a TEM00 mode can couple into higher order modes.  
	
	This can be broken up into two categories: mode misalignment and mode mismatch.  Everyone says it differently.
	
	Summarize the coupling [Anderson and Morrison]
	
	Summarize the formalism with ket notation as a sum of the eigenmodes of a cavity.
	References:[Anderson, Guido]
	
	
	\chapter{Bullseye Photodiode Characterization}
	
	\section{DC}
	\begin{equation}
	\begin{split}
	\text{Power} &= \int_{A}^{B} \abs{A_{00}}^2e^{\frac{-2r^2}{\omega_{0}^2}} 2\pi r dr\\
			&= -\abs{A_{00}}^2 \frac{\pi \omega_{0}^2}{2} e^{\frac{-2r^2}{\omega_{0}^2}} \biggr\rvert_A^B
	\end{split}
	\end{equation}
	
	\begin{equation}
	\text{P}_{in} = \text{Power} \biggr\rvert_0^{r_0} = \abs{A_{00}}^2 \frac{\pi \omega_{0}^2}{2} [1 - e^{\frac{-2r_0^2}{\omega_{0}^2}}]
	\end{equation}
	
	\begin{equation}
	\text{P}_{out} = \text{Power} \biggr\rvert_{r_0}^{\infty} = \abs{A_{00}}^2 \frac{\pi \omega_{0}^2}{2} [e^{\frac{-2r_0^2}{\omega_{0}^2}}]
	\end{equation}
	
	\begin{equation}
	\text{P}_{total} =  \text{P}_{in} + \text{P}_{out}
	\end{equation}
	
	\begin{equation}
	\omega = \sqrt{\frac{\text{P}_{total}}{\abs{A_{00}}^2 \pi / 2}}
	\end{equation}
	
	\begin{equation}
	\text{DC Power Ratio} 
	= \frac{P_{out}}{P_{in}} \\
	= \frac{e^{-2r_0^2/ \omega_{0}^2}} {1 - e^{-2r_0^2/ \omega_{0}^2 }} \approx 0.582
	\end{equation}
	
	\section{RF}
	
	\begin{equation}
	\begin{split}
	\text{P}_{RF} &= \int_{A}^{B} \abs{A_{01}}^2 \big(1-\frac{2r^2}{\omega_{0}^2}\big) e^{\frac{-2r^2}{\omega_{0}^2}} 2\pi r dr\\
	&= -\abs{A_{00}}^2 \frac{\pi}{2} \omega_{0}^2 e^{\frac{-2r^2}{\omega_{0}^2}} \bigg(1 + \frac{4 r_4}{\omega_{0}^4} \bigg)  \biggr\rvert_A^B
	\end{split}
	\end{equation}
	
	\begin{equation}
	\text{P}_{in} \\
	= \text{P}_{RF} \biggr\rvert_0^{r_0} \\
	= - \abs{A_{01}}^2  \frac{\pi}{2} \omega_{0}^2 \bigg( e^{\frac{-2r^2}{\omega_{0}^2}} \big(1 + \frac{4 r_4}{\omega_{0}^4} \big) - 1 \bigg)
	\end{equation}
	
	\begin{equation}
	\text{P}_{out} \\
	= \text{P}_{RF} \biggr\rvert_{r_0}^{\infty}\\ 
	= - \abs{A_{01}}^2  \frac{\pi}{2} \omega_{0}^2 e^{\frac{-2r^2}{\omega_{0}^2}} \big(1 + \frac{4 r_4}{\omega_{0}^4} \big) 
	\end{equation}
	
	\begin{equation}
	\text{RF Power Ratio} 
	= \frac{P_{out}}{P_{in}} \\
	= \frac{e^{-2r_0^2/ \omega_{0}^2}} {1 - e^{-2r_0^2/ \omega_{0}^2 }} \approx 2.7844
	\end{equation}
	
	\chapter{Overlap of Gaussian Beams}
	
	Referenced in section
	
	The full Gaussian beam overlap is important in quantitatively defining the amount of power loss obtained when a cavity is mismatched a incoming laser field.
	
	First we define an arbitrary Gaussian beam in cylindrical coordinates:
	
	\begin{equation}
	\begin{split}
	\ket{A(r)} 
	&= \frac{A_0}{q(z)} e^{\frac{-ikr^2}{2q(z)}}\\
	&= \frac{A_0}{q(z)} e^{\frac{-ikr^2(z-iz_0)}{2\abs{q(z)}^2}}
	\end{split}
	\end{equation}
	
	where $A_0$ is a real amplitude, $q(z)= z + i z_0$ is the complex beam parameter, $k$ is the wave number, and $r$ is the radial variable in the transverse direction.

	First we normalize the overlap integral to unity:
	\begin{equation}
	\braket{A(r)|A(r)} 
	=  \frac{\rvert A_0 \rvert^2}{z^2+z_0^2} \int_{0}^{\infty} e^{\frac{-kr^2 z_0}{\abs{q(z)}^2}} 2 \pi r dr = 1
	\end{equation}

	Normalization factor is this:
	\begin{equation}
	A_0 = \sqrt{\frac{k z_0}{\pi}}
	\end{equation}

	For two Gaussian beams with arbitrary q-parameters
	\begin{equation}
	\ket{A_i} = \frac{A_{0,i}}{q_i} e^{ \frac{-ikr^2(z-iz_0)}{2\abs{q_i}^2 }}
	\end{equation}
	where $z_{0,i}$ is the waist size of one particular beam.
	
	The power overlap is:
	\begin{equation}
	\text{Overlap} = \braket{A_1|A_2} = 4 \frac{ \sqrt{z_{0,1}z_{0,2}}}{\abs{q_1 - q_2^*}^2}
	\end{equation}

\end{appendices} 

\medskip

\bibliographystyle{unsrt}
\bibliography{Refs}



\end{document}