\chapter{Mode Matching Cavities at LIGO Hanford}
	Reference[Dan Hoake, Kognelik and Li, aLOGS]
	
	Modeling: Finesse vs Alamode
	
	Measurements: Mode profiling and Cavity Scanning
	
	Misalignments described in Chapter \ref{fund_mm} can be actively suppressed using the modal decomposition technique which leaves reduces the amount of odd order modes present in the cavity.  This leaves the even order modes which arise from modal mismatch to be the dominate sources of optical losses assuming mirrors with a small amount of scatter loss($<100$ ppm).

	
	\section{SRC Gouy Phase Measurement}
	The importance of mode matching actually goes beyond reducing the amount of losses in coupled cavities.  It also is important for cavity stability.  If we look at the g-factor of a cavity, it is required through ABCD transformations that the values lay between 0 and 1.  For the signal recycling cavity, if the round trip gouy phase is off by a few millimeters, the stability of the cavity can be compromised. 
	
	\subsection{SR3 Heater}
	Beckhoff code
	

	\subsection{SRM Heater}
	Test set up in OSB optics lab
	
	
	\section{Mode Matching IFO to OMC: Single Bounce vs DRMI}
	- Mode Profiling: Single Bounce
	- OMC Scanning
	
	
	\section{Mode Matching Squeezer to OMC}
	- Mode Profiling
	- OMC Scanning
	- Astigmatism: expected in LIGO and measured!

	\section{Modal Contrast Defect}
	
	\section{Mode Matching as a Function of DARM Offset}
	
	\section{Hot vs. Cold Interferometers}
	- Kilowatts of arm power
	- 